\begin{table}
  \begin{center}
    \caption{Result of how requirements were tackled.}
    \begin{tabular}{| l | p{12cm} |}
      \hline
        \textbf{Requirement} &
        \textbf{CloudML solution} \\
      \hline
        \citereq{software-reuse} & The engine in CloudML include and rely heavily 
                                   on an external library.
                                   This library is used to interface cloud providers
                                   with the engine.
                                   In the end this solution prevents CloudML from
                                   \emph{``reinventing the wheel''}, and keep focus 
                                   on the task at hand.
                                   \\ \hline
        \citereq{foundation} & The implementation of CloudML is developed in Scala.
                               This language introduces \emph{``state-of-the-art''}
                               features, while at the same time leverage support for
                               software industry through the \myac{JVM}.\\ \hline
        \citereq{mda} & In CloudML a model-based meta-model is designed.
                        The solution also let end users design templates with models. \\ \hline
        \citereq{lexical-template} & The engine is capable of parsing and interpreting templates.
                                     Then it configure and provision instances based on these templates.
                                     These templates are lexical, in the form of \myac{JSON}.  \\ \hline
        \citereq{m@rt} & Because of long time intervals for node provisioning CloudML introduces
                         \myac{M@RT}. This is done through asynchronous behavior of actor model,
                                      which is built into the underlying technology, Scala.
                                      Then the actor model is combined with observer pattern,
                                      enhancing the engines ability to do asynchronous callbacks. \\ \hline
        \citereq{multi-cloud} & The library \emph{jclouds} is included in the engine,
                                giving it support for $24$ providers. \\ \hline
    \end{tabular}
  \end{center}
  \label{table:results}
\end{table}

