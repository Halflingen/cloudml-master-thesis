\begin{figure}
  \tikzstyle{class}=[rectangle, draw=black, rounded corners, fill=white!40, drop shadow,
  text centered, anchor=north, text=black, text width=3.5cm]
  \tikzstyle{m@rt}=[rectangle, draw=black, rounded corners, fill=gray!40, drop shadow,
  text centered, anchor=north, text=black, text width=3.5cm]

  \begin{center}
      \begin{tikzpicture}[scale=0.6, transform shape]
      \node (Item) [class, rectangle split, rectangle] {
        \textbf{UserLibrary}
      };
    \node (CloudMLEngine) [class, rectangle, rectangle split parts=3, below=of Item] {
            \textbf{CloudMLEngine}
            \nodepart{second}
            \nodepart{third}+build
      };
    \node (AuxNode01) [text width=4cm, below=of CloudMLEngine] {};

    \node (Account) [class, rectangle split, rectangle split parts=2, left=of CloudMLEngine] {
            \textbf{Account}
            \nodepart{second}+name: String
        };
    \node (Credential) [class, rectangle, rectangle, below=of Account] {
            \textbf{Credential}
        };
    \node (Password) [class, rectangle split, rectangle split parts=2, below=of Credential, xshift=-4cm] {
            \textbf{Password}
            \nodepart{second}+identity: String \\ ++credential: String
        };
    \node (KeyPair) [class, rectangle split, rectangle split parts=2, below=of Credential] {
            \textbf{KeyPair}
            \nodepart{second}+public: String
        };

    \node (Connector) [class, rectangle, rectangle, right=of CloudMLEngine] {
            \textbf{Connector}
        };
    \node (AmazonEC2) [class, rectangle, rectangle, below=of Connector, xshift=-2cm] {
            \textbf{AmazonEC2}
        };
    \node (Rackspace) [class, rectangle, rectangle, below=of Connector, xshift=2cm] {
            \textbf{Rackspace}
        };

    \node (System) [m@rt, rectangle, rectangle, below=of AuxNode01] {
            \textbf{System}
        };
    \node (RuntimeInstance) [m@rt, rectangle, rectangle, below=of System, yshift=-1cm] {
            \textbf{RuntimeInstance}
        };
    \node (RuntimeProp) [m@rt, rectangle, rectangle, left=of RuntimeInstance] {
            \textbf{RuntimeProp}
        };
    \node (PublicIP) [m@rt, rectangle split, rectangle split parts=2, below=of RuntimeProp, xshift=-2cm] {
            \textbf{PublicIp}
            \nodepart{second}+Value: Address
        };
    \node (PrivateIP) [m@rt, rectangle split, rectangle split parts=2, below=of RuntimeProp, xshift=2cm] {
            \textbf{PrivateIp}
            \nodepart{second}+Value: Address
        };

    \node (Template) [class, rectangle split, rectangle split parts=2, right=of System] {
            \textbf{Template}
            \nodepart{second}+name: String
        };
    \node (Node) [class, rectangle split, rectangle split parts=2, below=of Template] {
            \textbf{Node}
            \nodepart{second}+id: String
        };
    \node (Property) [class, rectangle, rectangle, below=of Node] {
            \textbf{Property}
        };
    \node (Location) [class, rectangle split, rectangle split parts=2, below=of Property] {
            \textbf{Location}
            \nodepart{second}+value: String
        };
    \node (Disk) [class, rectangle split, rectangle split parts=2, left=of Location] {
            \textbf{Disk}
            \nodepart{second}+min: String
        };
    \node (Core) [class, rectangle split, rectangle split parts=2, left=of Disk] {
            \textbf{Core}
            \nodepart{second}+min: String
        };
    \node (RAM) [class, rectangle split, rectangle split parts=2, left=of Core] {
            \textbf{RAM}
            \nodepart{second}+min: String
        };
    \end{tikzpicture}
  \end{center}
  \caption{Architecture of CloudML}
  \label{fig:architecture}
\end{figure}
