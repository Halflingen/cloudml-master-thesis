\begin{figure}
  \begin{center}
    \subfigure[Template with nodes] {
      \begin{tikzpicture}[scale=0.7, transform shape]
        \node (Template) [class, rectangle split, rectangle split parts=2] {
          \textbf{:Template}
          \nodepart{second}name=``template1''
        };
        \node (Node02) [class, rectangle split, rectangle split parts=2, right=of Template] {
          \textbf{:Node}
          \nodepart{second}name=``node2'' \\ cores=2
        };
        \node (Node01) [class, rectangle split, rectangle split parts=2, above=of Node02] {
          \textbf{:Node}
          \nodepart{second}name=``node1'' \\ cores=2
        };
        \node (Node03) [class, rectangle split, rectangle split parts=2, below=of Node02] {
          \textbf{test1:Node}
          \nodepart{second}name=``node3'' \\ disk=2000
        };

        \draw[line] (Template) -- (Node01);
        \draw[line] (Template) -- (Node02);
        \draw[line] (Template) -- (Node03);
      \end{tikzpicture}
      \label{fig:scenario2-1}
    }

    \subfigure[Instance] {
      \begin{tikzpicture}[scale=0.7, transform shape]
        \node (Instance01) [class, text width=4.5cm, rectangle split, rectangle split parts=2] {
          \textbf{:Instance}
          \nodepart{second}name=``node1'' \\ cores=2 \\ templateName=``template1''
        };
        \node (Instance02) [class, text width=4.5cm, rectangle split, rectangle split parts=2, right=of Instance01] {
          \textbf{:Instance}
          \nodepart{second}name=``node2'' \\ cores=2 \\ templateName=``template1''
        };
        \node (Instance03) [class, text width=4.5cm, rectangle split, rectangle split parts=2, right=of Instance02] {
          \textbf{:Instance}
          \nodepart{second}name=``node4'' \\ disk=2000 \\ templateName=``template1''
        };
      \end{tikzpicture}
      \label{fig:scenario2-2}
    }
  \end{center}
  \caption{Scenario1}
  \label{fig:scenario2}
\end{figure}

