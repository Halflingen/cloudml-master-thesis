\def\top{
  \newthread{u}{:User}
  \newthreadhack{c}{:CloudML}
  \newinst{r1}{r1:RI}
  \newinst{r2}{r2:RI}
  \newinst{r3}{r3:RI}
  \newthread{a}{:AWS}
}

\begin{figure}[tb]
  \centering
  \begin{sequencediagram}[scale=0.9, transform shape]
    \top
    \begin{call}{u}{build(\texttt{account},List(\texttt{template}))}{c}{
        List(\texttt{r1}, \texttt{r2}, \texttt{r3})}
    \begin{call}{c}{Initialize()}{r1}{}
    \end{call}
    \begin{call}{c}{Initialize()}{r2}{}
    \end{call}
    \begin{call}{c}{Initialize()}{r3}{}
    \end{call}
    \end{call}

    \begin{messcall}{c}{provision nodes(\texttt{r1}, \texttt{r2}, \texttt{r3})}{a}
    \end{messcall}
  \end{sequencediagram}
  \captcont{Three nodes provisioning: Provisioning.}
  \label{fig:sequence-threenodes-1}
\end{figure}
\begin{figure}[tb]
  \centering
  \begin{sequencediagram}[scale=0.9, transform shape]
    \top
    \begin{call}{u}{getStatus()}{r1}{\emph{"Building"}}
    \end{call}
    \begin{call}{u}{getStatus()}{r2}{\emph{"Building"}}
    \end{call}

    \begin{messcall}{a}{status(\texttt{r1}, \emph{"Starting"})}{c}
    \end{messcall}
    \begin{messcall}{c}{update(\emph{"Starting"})}{r1}
    \end{messcall}


    \begin{call}{u}{getStatus()}{r1}{\emph{"Starting"}}
    \end{call}
    \begin{call}{u}{getStatus()}{r2}{\emph{"Building"}}
    \end{call}

    \begin{messcall}{a}{status(\texttt{r1}, \emph{"Started"})}{c}
    \end{messcall}
    \begin{messcall}{c}{update(\emph{"Started"})}{r1}
    \end{messcall}
    \begin{messcall}{a}{status(\texttt{r2}, \emph{"Starting"})}{c}
    \end{messcall}
    \begin{messcall}{c}{update(\emph{"Starting"})}{r2}
    \end{messcall}

    \begin{call}{u}{getStatus()}{r1}{\emph{"Started"}}
    \end{call}
    \begin{call}{u}{getStatus()}{r2}{\emph{"Starting"}}
    \end{call}
    \begin{call}{u}{getStatus()}{r3}{\emph{"Building"}}
    \end{call}
  \end{sequencediagram}
  \captcont{Three nodes provisioning: Asynchronous message communication.}
  \label{fig:sequence-threenodes-2}
\end{figure}
\begin{figure}[tb]
  \centering
  \begin{sequencediagram}[scale=0.9, transform shape]
    \newthread{u}{:User}
    \newthreadhack{c}{:CloudML}
    \newinst{r1}{r1:RI}
    \newinst{r2}{r2:RI}
    \newinst{r3}{r3:RI} \newthread{a}{:AWS}

    \begin{messcall}{a}{status(\texttt{r3}, \emph{"Starting"})}{c}
    \end{messcall}
    \begin{messcall}{c}{update(\emph{"Starting"})}{r3}
    \end{messcall}
    \begin{messcall}{a}{status(\texttt{r2}, \emph{"Started"})}{c}
    \end{messcall}
    \begin{messcall}{c}{update(\emph{"Started"})}{r3}
    \end{messcall}
    \begin{messcall}{a}{status(\texttt{r2}, \emph{"Started"})}{c}
    \end{messcall}
    \begin{messcall}{c}{update(\emph{"Started"})}{r3}
    \end{messcall}

    \begin{messcall}{c}{createLoadBalancer(\texttt{r1}, \texttt{r2}, \texttt{r3})}{a}
    \end{messcall}
  \end{sequencediagram}
  \caption{Three nodes provisioning: Load balancer.}
  \label{fig:sequence-threenodes-3}
\end{figure}
