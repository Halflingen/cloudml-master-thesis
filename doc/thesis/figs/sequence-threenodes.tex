\def\topthreenodes {
  \newthread{u}{:User}
  \newthreadhack{c}{:CloudML}
  \newinst{r1}{r1:RI}
  \newinst{r2}{r2:RI}
  \newinst{r3}{r3:RI}
  \newinst{a}{:CC}
}

\begin{figure}[tb]
  \centering
  \begin{sequencediagram}[scale=0.8, transform shape]
    \topthreenodes
    \begin{call}{u}{build(\texttt{account},List(\texttt{template}))}{c}{
        List(\texttt{r1}, \texttt{r2}, \texttt{r3})}
      \begin{call}{c}{Initialize()}{r1}{}
      \end{call}
      \begin{call}{c}{Initialize()}{r2}{}
      \end{call}
      \begin{call}{c}{Initialize()}{r3}{}
      \end{call}
    \end{call}

    \begin{messcall}{c}{update(\emph{"Building"})}{r1}
    \end{messcall}
    \begin{messcall}{c}{update(\emph{"Building"})}{r2}
    \end{messcall}
    \begin{messcall}{c}{update(\emph{"Building"})}{r3}
    \end{messcall}

    \begin{call}{c}{buildTemplates(\texttt{r1}, \texttt{r2}, \texttt{r3})}{a}{}
    \end{call}

    \begin{messcall}{c}{update(\emph{"Starting"})}{r1}
    \end{messcall}
    \begin{messcall}{c}{update(\emph{"Starting"})}{r2}
    \end{messcall}
    \begin{messcall}{c}{update(\emph{"Starting"})}{r3}
    \end{messcall}
  \end{sequencediagram}
  \caption{Provisioning three nodes provisioning.}
  \label{fig:sequence-threenodes-1}
\end{figure}
\begin{figure}[tb]
  \centering
  \begin{sequencediagram}[scale=0.8, transform shape]
    \topthreenodes
    \begin{call}{u}{getStatus()}{r2}{\emph{"Starting"}}
    \end{call}

    \begin{call}{c}{provision(\texttt{r1})}{a}{}
    \end{call}
    \begin{messcall}{c}{update(\emph{"Started"})}{r1}
    \end{messcall}

    \begin{messcall}{c}{update(\emph{"Started"})}{r1}
    \end{messcall}


    \begin{call}{u}{getStatus()}{r1}{\emph{"Started"}}
    \end{call}
    \begin{call}{u}{getStatus()}{r2}{\emph{"Starting"}}
    \end{call}

    \begin{call}{c}{provision(\texttt{r2})}{a}{}
      \begin{call}{u}{getStatus()}{r2}{\emph{"Starting"}}
      \end{call}
    \end{call}
    \begin{messcall}{c}{update(\emph{"Started"})}{r2}
    \end{messcall}
    \begin{call}{u}{getStatus()}{r2}{\emph{"Started"}}
    \end{call}

    \begin{call}{u}{getStatus()}{r1}{\emph{"Started"}}
    \end{call}
    \begin{call}{u}{getStatus()}{r3}{\emph{"Starting"}}
    \end{call}
    \begin{messcall}{c}{createLoadBalancer(\texttt{r1}, \texttt{r2}, \texttt{r3})}{a}
    \end{messcall}
  \end{sequencediagram}
  \caption{Asynchronous message communication between three nodes.}
  \label{fig:sequence-threenodes-2}
\end{figure}
