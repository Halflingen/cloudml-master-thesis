\documentclass[a4paper,norsk,11pt,twoside]{article}
\usepackage[latin1]{inputenc}
\usepackage[T1]{fontenc}
\usepackage[norsk]{babel}
\usepackage{epsfig}
\usepackage{graphicx}
\usepackage{amsmath}
\usepackage{pstricks}
\usepackage{subfigure}


\date{DATO}
\title{DOKUMENTTITTEL}
\author{FORFATTER}


% Kommentarer er markert med "%" i margen. Alle disse kan, om du
% �nsker det, fjernes i sin helhet.
%
% Erstatt ``DOKUMENTTITTEL'' med dokumentets tittel.
% Erstatt ``FORFATTER'' med ditt navn.
% Hvis du vil ha dagens dato hver gang du redigerer dokumentet kan du
% bytte ut DATO med \today, ellers erstatter du "DATO" med en fornuftig dato.
% Det finnes ogs� ved universitetet en pakke som heter "uioforside", denne kan du lese mer om her.


\begin{document}
\maketitle
\newpage


\tableofcontents{}
% denne kommandoen gir deg innholdsfortegnelse, s�fremt du har brukt \section{} og ikke
\section*{} (alts� at du har valgt nummererte avsnitt).


\section{AVSNITTSOVERSKRIFT}
% \section{} gir avsnitt
\subsection{UNDER-AVSNITTSOVERSKRIFT}
% \subsection{} gir under-avsnitt


% Punkt-liste:
\begin{itemize}
\item{} TEKST
\end{itemize}


% Nummerert punktliste:
\begin{enumerate}
\item{}
\end{enumerate}


% Sette inn bilde/figur:
\begin{figure}[hbt]
\begin{center}
\fbox{\includegraphics[width=\textwidth]{BILDE.eps}}
\caption{BILDEUNDERTEKST}\label{fig:finfigur}}
\end{center}
\end{figure}
% width = \textwidth gj�r at bildet blir like stort som teksten, her kan man endre st�rrelsen p� bildet ved � skrive "width = 10cm" f.eks.
%"[hbt]" gj�r at du pr�ver � overstyre LaTeX til � putte figuren HER i teksten (h), hvis ikke det g�r (figurer i LaTeX er flytende objekter, og vil gjerne ikke puttes seg akkurat der vi vil...), pr�ver vi � f� LaTeX til � sette figuren p� B�NN av siden (b), hvis ikke det g�r, s� p� TOPPEN (t)
% "\fbox{}" lager en ramme rundt figuren.
% Latex tar kun .eps-filer (og .ps-filer). Stort sett alle bildeformater kan konverteres til .eps i bildebehandlingsprogrammet Gimp.
% "\caption" er bildeteksten, hvis du vil ha teksten mindre kan man slenge p� "\caption{\small{tekst}}".
% "\label{}" er figurens "n�kkel". Denne n�klen kan du referere til hvor som helst i teksten din, referansen m� da se slik ut: "Se figur \ref{fig:figur1}".
% Latex m� f� kompilere to ganger for at den skal f� med seg b�de at det er en referanse, og for � kople referanse og label.


% Sette inn formel/matteelement (nummerert):
\begin{equation}\label{eq:9.141}
\varrho^{j}_{i}(t) =
\sqrt{(X^{j}(t)-X_{i}) +(Y^{j}(t)-Y_{i}) +(Z^{j}(t)-Z_{i}) } \equiv f(X_{i}, Y_{i}, Z_{i})
\end{equation}
% "\varrho" gir den greske bokstaven rho
% "\sqrt{}" gir kvadratrot
% "^{}" betyr at det som st�r inni {} skal opph�yes, "_{}" betyr det motsatte
% likningen blir seendes slik ut (st�r nederst p� siden)
% "\equiv" gir likhetstegn med tre streker
% equation lager nummererte formler, man kan ogs� bruke "$formel$" hvis man vil ha matteelementer midt i en tekst (alts� matteelementer mellom to dollartegn $$).


% Sette in en tabell:
\begin{table}
\begin{tabular}{|c|c|c|c|}
\hline
data & data & data & data
\hline
data & data & data & data
hline
data & data & data & data
\hline
\end{tabular}
\caption{TABELLFORKLARINGSTEKST}\label{tab:fintabell}
\end{table}
% "|c|c|c|c|" lager en tabell med fire kolonner
% "\hline" lager en horisontal linje
%du kan lage s� mange rader du vil, hver kolonne i raden skilles med &-tegn.
% "\caption" og "\label" er det samme som p� figur. Man m� ikke skrive "{tab:fintabell}", man kan skrive hva man vil, men hvis en skriver et stort dokument kan det v�re greit � skille mellom referanser som er fra tabeller, likninger, figurer og overskrifter.


% Referanseliste fra BibTeX:
\bibliography{referanser}
\bibliographystyle{norplain}
% "\bibliography{referanser}" henter inn filen "referanser.bib" hvor du har listet opp alle referansene dine.
%"\bibliographystyle{norplain}" forteller LaTeX at referanselisten din skal ha stilen "norplain"


\end{document} 