\mychapter{validation}{Validation \& Experiments}

\todo{
  \begin{itemize}
    \item How BankManager proves concepts of the templates (subsection 1) with cloudml-engine
  \end{itemize}
  Uncomment text
}

\section{Comparisons}

\todo{Moved here from implementation.}

Comparing challenges with some selected providers and technologies from \citechap{state-of-the-art}.

\begin{table}
    %\begin{tabular}{ | p{2cm} | p{2cm} | p{2.5cm} | p{2cm} | p{2cm} | p{2cm} |}
  \begin{tabular*}{\textwidth}{@{\extracolsep{\fill}}| l | l | l | l | l | l |}
    %\begin{tabular*}{\textwidth}{ | l | l | l | l | l | l |}
      \hline
        \textbf{State of the art} & 
        \textbf{\citereq{software-reuse}} & 
        \textbf{\citereq{foundation}} & 
        \textbf{\citereq{mda}} & 
        \textbf{\citereq{m@rt}} & 
        \textbf{\citereq{lexical-template}} \\
      \hline
     Amazon CloudFormation & No & Hard & No & No & No \\ \hline
     CA Applogic & Yes & Easy & Yes & N & No  \\ \hline
     Libcloud & No & Hard & No & Yes & No \\ \hline
     jclouds & No & Hard & No & Yes & No \\ \hline
     OPA & Yes & Hard & No & No & No \\ \hline
     Whirr & No & Hard & No & Yes & No \\ \hline
     Deltacloud & No & Hard & No & Yes& No  \\ \hline
     CloudML & Yes & Easy & Yes & Yes & No \\ \hline
  \end{tabular*}
  \caption{Analysis}
  \label{table:analysis}
\end{table}




To validate how CloudML addressed the requirements from~\citechap{requirements},
the \emph{BankManager} application were provisioned using different topologies~(Fig[\ref{fig:singlenode},~\ref{fig:threenodes}]).
The implementation uses \myac{JSON} to define templates as a human readable serialization mechanism.
The lexical representation of \citefig{singlenode} can be seen in \citelist{singlenode}. 
The whole text represents the \texttt{Template} of \citefig{architecture} and consequently 
``nodes'' is a list of \texttt{Node} from the model.
The JSON is textual which makes it \emph{shareable} as files.
Once such a file is created it can be reused (\emph{reproducibility}) 
on any supported provider (\emph{multicloud}).

%\begin{lstlisting}[
  %language=javascript,
  %label=list:singlenode,
  %caption=One single node]
%{ "nodes": [ { "name": "testnode" } ] }
%\end{lstlisting}

The topology described in \citefig{threenodes} is represented in \citelist{threenodes},
the main difference from \citelist{singlenode} is that there are two more nodes and a total of 
five more properties.
Characteristics of each node are carefully chosen based on each nodes feature area, for instance 
front-end nodes have more computation power, while the back-end node will have more disk.

%\begin{lstlisting}[
  %language=javascript,
  %label=list:threenodes,
  %caption=Three nodes]
%{
  %"nodes": [ 
    %{ "name": "frontend1", "minRam": 512, "minCores": 2 },
    %{ "name": "frontend2", "minRam": 512, "minCores": 2 },
    %{ "name": "backend", "minDisk": 100 }
  %]
%}
%\end{lstlisting}
