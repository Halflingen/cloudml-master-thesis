\mychapter{validation}{Validation \& Experiments}
% \begin{figure}
  \begin{sequencediagram}
    \newthread{u}{:User}
    \newthreadhack{c}{:CloudML}
    \newinst{r}{:RuntimeInstance}
    \newthread{a}{:AWS}

    \begin{call}{u}{build(\texttt{account},List(\texttt{template}))}{c}{List(\texttt{RuntimeInstance})}
      \begin{call}{c}{Initialize()}{r}{}
      \end{call}
    \end{call}
    \begin{messcall}{c}{provision node}{a}
    \end{messcall}

    \begin{call}{u}{getStatus()}{r}{\emph{Building}}
    \end{call}

    \begin{messcall}{a}{property update}{c}
    \end{messcall}

    \begin{call}{u}{getStatus()}{r}{\emph{Starting}}
    \end{call}
  \end{sequencediagram}

  \caption{Sequence diagram of CloudML.}
  \label{fig:sequencediagram}
\end{figure}


\todo{
  \begin{itemize}
    \item How BankManager proves concepts of the templates (subsection 1) with cloudml-engine
  \end{itemize}
}


To validate how CloudML addressed the challenges from~\citetable{challenges}
we provisioned the \emph{BankManager} application using different topologies in~Fig[\ref{fig:singlenode},~\ref{fig:threenodes}].
The implementation uses \emph{JavaScript Object Notation}~(JSON) to define templates
as a human readable serialization mechanism.
The lexical representation of \citefig{singlenode} can be seen in \citelist{singlenode}. 
The whole text represents the \texttt{Template} of \citefig{architecture} and consequently 
``nodes'' is a list of \texttt{Node} from the model.
The JSON is textual which makes it \emph{shareable} as files.
We implemented it so once such a file is created it can be reused (\emph{reproducibility}) 
on any supported provider (\emph{multicloud}).

\begin{lstlisting}[
  language=javascript,
  label=list:singlenode,
  caption=One single node]
{ "nodes": [ { "name": "testnode" } ] }
\end{lstlisting}

The topology described in \citefig{threenodes} is represented in \citelist{threenodes},
the main difference from \citelist{singlenode} is that there are two more nodes and a total of 
five more properties.
Characteristics of each node are carefully chosen based on each nodes feature area, for instance 
front-end nodes have more computation power, while the back-end node will have more disk.

\begin{lstlisting}[
  language=javascript,
  label=list:threenodes,
  caption=Three nodes]
{
  "nodes": [ 
    { "name": "frontend1", "minRam": 512, "minCores": 2 },
    { "name": "frontend2", "minRam": 512, "minCores": 2 },
    { "name": "backend", "minDisk": 100 }
  ]
}
\end{lstlisting}
