\chapter{State of the Art in Provisioning}

\note{
Evaluation of existing solutions \\
What have others done for multicloud provisioning \\
Even more examples in mOSAIC articles
}

\todo{
  \begin{itemize}
    \item Identify \emph{properties} (problems in reality)
    \item Find more sources
  \end{itemize}
}

\section{Model driven}

\paragraph{Amazon AWS CloudFormation}
\url{http://aws.amazon.com}

This is a service provided by Amazon from their popular Amazon Web Services. 
It give users the ability to create template files in form of JSON, which they can load into AWS to create stacks of resources. 
This makes it easier for users to duplicate a setup many times, and as the templates support parameters this process 
can be as dynamic as the user design it to be. This is a model in form or lexical syntax, both the template itself and the resources that can be used.
For a company that is fresh in the world of cloud computing this service could be considered too advance. 
This is mainly meant for users that want to replicate a certain stack, with the ability to provide custom parameters. 
Once a stack is deployed it is only maintainable through the AWS Console, and not through template files. 
The format that Amazon uses for the templates is a very good format, the syntax is in form of JSON which is very readable and easy to use, 
but the structure and semantics of the template itself is not used by any other providers or cloud management tooling, 
so it can not be considered a multicloud solution. Even though JSON is a readable format, does not make it viable as a presentation medium on a business level.

\paragraph{CA Applogic}
\url{http://www.3tera.com/AppLogic/}

Applogic from CA is a proprietary model based web application for management of private clouds. 
This interface let users configure their deployments through a diagram with familiarities to component diagrams with interfaces and assembly connectors. 
This is one of the solutions that use and benefit from a model based approach. They let users configure a selection of third party applications, 
such as Apache and MySQL, as well as network security, instances and monitoring. 
What CA has created is both an easy way into the cloud and it utilizes the advantages of model realizations. 
Their solution will also prove beneficial when conducting business level consulting. 
They also support a version of ADL (Architecture Deployement Language), a good step on its way to standardization. 
But this solution is only made for private clouds running their own controller, this can prove troublesome for migration, both in to and out of the infrastructure.

\section{APIs}

\paragraph{libcloud and jclouds}
\url{http://libcloud.apache.org/}
\url{http://www.jclouds.org/}

Libcloud is a API that aims to support the largest cloud providers through a common API. 
The classes are based around "Drivers" that extends from a common ontology, then provider-specific attributes and logic is added to the implementation.
jclouds is very similar to libcloud but the API code base is written in Java and Clojure. 
This library also have "drivers" for different providers, but they also support some PaaS solutions such as Google App Engine.
APIs can be considered modelling approaches based on the fact they have a topology and hierarchical structure, 
but it is not a distinct modelling. A modelling language could overlay the code and help providing a clear overview, 
but the language directly would not provide a good overview of deployment. 
And links between resources can be hard to see, as the API lacks correlation between resources and method calls. 
Libcloud have solved the multicloud problem in a very detailed manner, but the complexity is therefore even larger. 
The API is also Python-only and could therefor be considered to have high tool-chain dependency.

\paragraph{OPA}
\url{http://opalang.org/}

OPA is a cloud language aimed at easing development of modern web applications. It is a new language, 
with its own syntax, which is aimed directly at the web. The language will build into executable files that will handle load balancing and scalability, 
this is to to make this a part of the language and compilation.
OPA is a new language, so it might be difficult to migrate legacy systems into this lanugage. 
There are no deployment configurations, as this is built into the language. The compiler will generate an executable that coWeb-based vs native application

The public cloud is located on the world wide web, and most of the managing, monitoring, 
payment and other administrative tasks can be done through web interfaces or APIs. 
Web applications are becoming more popular by the day, with HTML5, EcmaScript 5 and CSS3. 
The user experience in web applications today can in many cases match native applications, with additional benefits such as availability and ease of use.
A web-based interface would prove beneficial for quickly displaying the simple core functionality of the language. 
In this era of cloud computing and cloud technologies a user should not need to abandon his or hers browser to explore the functionality of CloudML.
Cloud providers are most likely to give customers access to customize their cloud services through web-based interfaces, 
and if customers are to take advantage of CloudML, the language should be graphically integrated into existing tool chains. 
Providers would probably find it pleasing if a example GUI wauld be run on most cloud providers instances, 
and so it can also benefit from some cloud based load balancers, even though this is part of the language. 
The conclusion about OPA is that it is not a language meant for configuration, and could not easily benefit from a model based approach, 
and it does not intentionally solve multicloud.

\paragraph{Whirr}
\url{http://whirr.apache.org/}

This is a binary and code-based application for creating and starting short-lived clusters for hadoop instances.
It support multiple cloud providers. It has a layout for configuration but it is mainly property-based, and aimed at making clusters. 

\paragraph{Deltacloud}
\url{http://incubator.apache.org/deltacloud/}

Deltacloud has a similar procedure as jclouds and libcloud, but with a REST API. 
So they also work on the term "driver", but instead of having a library to a programming language the users are presented with an API they can call, 
on Deltacloud servers. This means users can write in any language they may choose. 
As well as having similar problems as other APIs this approach means that every call has to go through their servers, 
similar to a proxy. This can work with the benefits that many middleware software have, such as cahing, queues, 
redundancy and transformations, but it also has the disadvantages such as single point of failure and version inconsistencies.

\section{Deployments}

\paragraph{Amazon Beanstalk}
\paragraph{simplifying-solution-deployment-on-a-cloud-through-composite-appliances}
\paragraph{architecture-for-virtual-solution-composition-and-deployment}

