\chapter{Introduction}

Cloud Computing~\cite{Armbrust:EECS-2009-28} is gaining in popularity,
both in the academic world and in software industry.
One of the greatest advantages of Cloud Computing is \emph{on-demand self service},
providing dynamic scalability~(\emph{elasticity}).
As Cloud Computing utilizes \emph{pay-as-you-go} payment solution,
customers are offered fine-grained costs for rented services.
These advantages can be exploited to prevent web-applications
from breaking during peak loads, and supporting an \emph{``unlimited''}
amount of end users.
According to Amazon (provider of \myac{AWS}, a major Cloud Computing host):
\emph{``much like plugging in a microwave in order to power it doesn't require any knowledge of electricity,
one should be able to plug in an application to the cloud in order
to receive the power it needs to run, 
just like a utility''}~\cite{aws:varia10}.
%Although the technical challenge when ``plugging'' an application into the cloud
%is yet transparent,
Although there are still technical challenges when deploying applications to the cloud,
as Amazon emphasizes them selves.

Cloud providers stresses the need of technological advantages originating from Cloud Computing.
They emphasize how horizontal and vertical scalability can enhance applications
to become more robust.
The main issue is technological inconsistencies for provisioning between providers,
\eg \myac{AWS} offer \myac{CLI} tools, while Rackspace only offer web-based \myac{API}s.
Although most providers offer APIs these are different as well, 
they are inconsistent in layout, usage and entities.
The result of this is \emph{vendor lock-in}, the means used to provision an application to
a given cloud must be reconsidered if it is to be re-provisioned on another provider.

This thesis introduces the first version of CloudML,
a modeling language designed to harmonize the inconsistencies between providers,
with a model-based approach.
This research is done in the context of the REMICS EU FP7 project,
which aims to provide automated support to migrate legacy applications into
clouds~\cite{DBLP:conf/services/MohagheghiS11}.
With CloudML users can design cloud topologies with models,
and while provisioning they are provided \emph{``run-time models''} of 
resources under provisioning,
according to models@run.time approach~\cite{DBLP:journals/dagstuhl-reports/AssmannBCF11}.

Accompanying this thesis is a research into \emph{``state-of-the-art''} technologies and frameworks.
This background information give an overview, 
indication the position of cloud evolution today~(\date{April 2012}).
An experiment is conducted to outline the challenges with cloud provisioning today.
From these challenges a set of requirements are constructed, which will be
referred to throughout the thesis.
The goal of CloudML is to address these requirements,
and this is approached through visioning, designing and implementing CloudML.
Finally an experiment is carried out to validate the implementation of CloudML.

\paragraph{Publications.}

\begin{itemize}
  \item
    Paper presented at BENEVOL'11:
    \bibentry{mosser-brandtzæg-etal:2011}.
    This paper introduces the motivation of the global approach for CloudML.
  \item
    Paper accepted at Cloud'12:
    \bibentry{cloud12}.
    This paper state the deployment procedure of \emph{PIM4Cloud}.
    Deployment is the next step after provisioning,
    a future enhancement to CloudML.
  \item 
    Technical report for Deliverable D4.1:
    \bibentry{cloud12}.
    This report describe a global overview of the migration process of \myac{REMICS}.
  \item
    Paper submitted to Cloud the MDE workshop, associated with the \myac{ECMFA}:
    \bibentry{ecmfa4clouda}.
    This paper introduces the design and early requirements of CloudML.
\end{itemize}

\paragraph{Involvement in REMICS.}

\myac{REMICS} is a $5,7$ M {\euro} project, and part of the $7th$ Framework Program.
Its main goal is to migrate legacy application,
\eg software written in COBOL or old versions of Pascal,
to modern web-based application for the cloud.
Some of the partners in \myac{REMICS} are SINTEF, DOME consuling, 
Tecnalia, Fraunhofer FOKUS, Netfective, DI Systemer and SOFTEAM.
My involvement in \myac{REMICS} is CloudML,
which focuses on the last step in the process of this goal,
which is deploying these application to cloud environments.

I have been involved in \myac{REMICS} Deliverable D4.1 (PIM4Cloud, Work package 4).
I have attended two project meetings consulting PIM4Cloud and respectively CloudML.
The first one at DOME consulting, Palma de Mallorca~(\date{June 2011}).
The second project meeting by traveling with \emph{Hurtigruten} from Troms{\o} to Trondheim~(\date{September 2011}).
