\chapter{Introduction}

Cloud Computing~\cite{Armbrust:EECS-2009-28} is gaining in popularity,
both in the academic world and in software industry.
One of the greatest advantages of Cloud Computing is \emph{on-demand self service},
providing dynamic scalability~(\emph{elasticity}).
As Cloud Computing utilizes \emph{pay-as-you-go} payment solution,
customers are offered fine-grained costs for rented services.
These advantages can be exploited to prevent web-applications
from breaking during peak loads, and supporting an \emph{``unlimited''}
amount of end users.
According to Amazon (provider of \myac{AWS}, a major Cloud Computing host):
\emph{``much like plugging in a microwave in order to power it doesn't require any knowledge of electricity,
one should be able to plug in an application to the cloud in order
to receive the power it needs to run, 
just like a utility''}~\cite{aws:varia10}.
%Although the technical challenge when ``plugging'' an application into the cloud
%is yet transparent,
Although there are still technical challenges when deploying applications to the cloud,
as Amazon emphasizes them selves.

Cloud providers stresses the need of technological advantages originating from Cloud Computing.
They emphasize how horizontal and vertical scalability can enhance applications
to become more robust.
The main issue is technological inconsistencies for provisioning between providers,
\eg \myac{AWS} offer \myac{CLI} tools, while Rackspace only offer web-based \myac{API}s.
Although most providers offer APIs these are different as well, 
they are inconsistent in layout, usage and entities.
The result of this is \emph{vendor lock-in}, the means used to provision an application to
a given cloud must be reconsidered if it is to be re-provisioned on another provider.

This thesis introduces the first version of CloudML,
a modeling language designed to harmonize the inconsistencies between providers
with a model-based approach.
This research is done in the context of the REMICS EU FP7 project,
which aims to provide automated support to migrate legacy applications into
clouds~\cite{DBLP:conf/services/MohagheghiS11}.
With CloudML users can design cloud topologies with models,
and while provisioning they are provided \emph{``run-time models''} of 
resources under provisioning,
according to models@run.time approach~\cite{DBLP:journals/dagstuhl-reports/AssmannBCF11}.



\paragraph{Publications.}

In context of this thesis a paper was written and submitted,
there is also two additional papers originating from the same research.

\begin{itemize}
  \item
    Paper submitted to Cloud MDE workshop, associated with the \myac{ECMFA}:
    \bibentry{ecmfa4clouda}.
  \item
    Paper accepted at Cloud'12, involvement in REMICS (Deliverable D4.1):
    \bibentry{remics_4_1}.
  \item
    Paper presented at BENEVOL'11:
    \bibentry{mosser-brandtzæg-etal:2011}.
   \item 
     Deliverable D4.1
\end{itemize}

\paragraph{Involvement in REMICS.}

I, the author, have been involved in \myac{REMICS} Deliverable D4.1 (PIM4Cloud, Work package 4).
I have attended two workshop meetings consulting PIM4Cloud and respectively CloudML.
The first one at DOME consulting, Palma de Mallorca~(\date{June 2011}).
The second workshop by traveling with \emph{Hurtigruten} from Troms{\o} to Trondheim~(\date{September 2011}).
Other partners attending the workshops, beside SINTEF, are DOME, Tecnalia, Fraunhofer FOKUS, Netfective, DI Systemer and SOFTEAM.
