\mychapter{implementation}{Implementation/realization - cloudml-engine}
\begin{figure}
  \tikzstyle{module}=[rectangle, draw=black, rounded corners, fill=white!40, drop shadow,
  text centered, anchor=north, text=black, text width=3.5cm, top color=white, bottom color=black!10]
  \tikzstyle{arrow}=[->, >=triangle 60]
  \tikzstyle{flow}=[->, >=open triangle 90]

  \begin{center}
    \begin{tikzpicture}[scale=1, transform shape]
      \node (AuxNode01) {};
      \node (Engine) [module, left=of AuxNode01, rectangle split, rectangle split parts=2] { 
        \textbf{Engine} 
        \nodepart{second}Entrypoint. Orchestration.
      };
      \node (AuxNode02) [left=of Engine] {};
      \node (Kernel) [module, below=of AuxNode01, rectangle split, rectangle split parts=2] { 
        \textbf{Kernel} 
        \nodepart{second}Node domains. Converts JSON to Node Entities.
      };
      \node (Repository) [module, above=of AuxNode01, rectangle split, rectangle split parts=2] { 
        \textbf{Repository} 
        \nodepart{second}Instance domains. Convert Nodes to Instances.
      };
      \node (Cloud-Connector) [module, right=of AuxNode01, rectangle split, rectangle split parts=2] { 
        \textbf{Cloud-Connector} 
        \nodepart{second}Connects to providers (jclouds).
      };

      \draw [arrow] (Engine) -- (Kernel);
      \draw [arrow] (Engine) -- (Cloud-Connector);
      \draw [arrow] (Engine) -- (Repository);
      \draw [arrow] (Cloud-Connector) -- (Kernel.east);
      \draw [arrow] (Cloud-Connector) -- (Repository.east);
      \draw [arrow] (Repository) -- (Kernel);
      \draw [flow] (AuxNode02) -- (Engine);
    \end{tikzpicture}
  \end{center}
  \caption{Architecture of cloudml-engine}
  \label{fig:cloudmlengine}
\end{figure}


\todo{
  \begin{itemize}
    \item More info than CloudMDE
    \item Technologies chosen
    \item Why technologies were chosen
  \end{itemize}
}

The envisions and designs of CloudML is implemented as a proof-of-concept project
(from here known as cloudml-engine).
Cloudml-engine is written in Scala, a multi-paradigm JVM based programming language.
This language was chosen because JVM is a popular platform, and then especially Java.
Scala is compatible with Java and Java can interact with libraries written in Scala as well.
The reason not to use plain Java was because Scala is an appealing state-of-the-art language that emphasizes 
on functional programming which is leveraged in the implementation.
Scala also has a built in system for Actors model~\cite{actors:haller07} which is utilized in the implementation.
For the lexical representation of CloudML \emph{JavaScript Object Notation}~(JSON) was chosen.
JSON is a web-service friendly, human-readable data interchange format and an alternative to XML.
This format was chosen because of popularity in the cloud community \todo{source}
and its usage area as data transmit format between servers and web applications.
This means cloudml-engine
