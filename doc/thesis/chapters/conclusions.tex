\mychapter{conclusions}{Conclusions}

In this thesis four main parts have been presented.
First the background part introducing the domain of cloud computing and model-driven engineering.
Then the second part highlighting sets of technologies, frameworks, ideas and 
\myac{API}s which are currently used in the two domains.
Third the challenges with these solutions are stressed, as well as a set of \emph{requirements}
CloudML must fulfill to tackle these challenges.
Lastly, CloudML is presented, in three phases,
\begin{ii}
  \iitem vision,
  \iitem design and
  \iitem implementation.
\end{ii}

In the vision chapter the core idea of CloudML were introduced,
and even means to tackle \citereq{m@rt} were outlined through pure vision.
The design chapter stated how CloudML should be built up,
what the meta-model should look like,
what underlying technologies should be used.
All through a scenario where Alice performs provisioning.
In this chapter the means to tackle \citereq{m@rt} are reinforced through
the view of design.
The requirements of \citereq{foundation} and \citereq{software-reuse}
are addressed through what underlying technology to use and alternatives.
Lastly the implementation chapter outline how CloudML is implemented
as \emph{cloudml-engine}, and how this solution is built up.
Both \citereq{mda} and \citereq{lexical-template} are tackled in this chapter
by concretely choosing data format and syntax based on the design chapter.

\section{Results}
\begin{table}
    %\begin{tabular}{ | p{2cm} | p{2cm} | p{2.5cm} | p{2cm} | p{2cm} | p{2cm} |}
  \begin{tabular*}{\textwidth}{@{\extracolsep{\fill}}| l | l | l | l | l | l |}
    %\begin{tabular*}{\textwidth}{ | l | l | l | l | l | l |}
      \hline
        \textbf{State of the art} & 
        \textbf{\citereq{software-reuse}} & 
        \textbf{\citereq{foundation}} & 
        \textbf{\citereq{mda}} & 
        \textbf{\citereq{m@rt}} & 
        \textbf{\citereq{lexical-template}} \\
      \hline
     Amazon CloudFormation & No & Hard & No & No & No \\ \hline
     CA Applogic & Yes & Easy & Yes & N & No  \\ \hline
     Libcloud & No & Hard & No & Yes & No \\ \hline
     jclouds & No & Hard & No & Yes & No \\ \hline
     OPA & Yes & Hard & No & No & No \\ \hline
     Whirr & No & Hard & No & Yes & No \\ \hline
     Deltacloud & No & Hard & No & Yes& No  \\ \hline
     CloudML & Yes & Easy & Yes & Yes & No \\ \hline
  \end{tabular*}
  \caption{Analysis}
  \label{table:analysis}
\end{table}



The implementation is validated through an experiment where
it is physically executed against two providers, \myac{AWS} and Rackspace.
This experiment concludes the work put into CloudML at this point to be successful.

The requirements from \citechap{requirements} are compared against selected
technologies and frameworks from \citechap{state-of-the-art}.
These comparisons are expressed in \citetable{requirements-comparison}.

These are the ``\emph{key results}'' composed by this thesis:
\begin{description}
  \item[meta-model.]
    A meta-model was designed, with model-driven engineering in mind.
    The meta-model was presented through a scenario where ``\emph{Alice}''
    provision nodes based on two specific topologies.
    The first topology consisting of a single node, to exemplify the simplest topology.
    The second setup was based on three nodes, representing a common multi-tier
    topology for development purposes.
    Lastly a description of how a multi-cloud provision was presented, 
    expressing how this is intended with the meta-model.
  \item[Engine.]
    An engine capable of provisioning nodes on a set of supported cloud providers
    was implemented.
    The engine takes advantage of the automatic building system Maven,
    enhancing the support for internal and external modules.
    Through Maven the engine becomes increasingly adaptable for developers
    seeking to include or utilize CloudML.
    Internally the engine is split into four logical sub-modules.
    Scala is used as programming language for the engine,
    providing a \emph{state-of-the-art} set of features,
    \eg combination of object-oriented and functional programming.
  \item[Lexical templates.]
    The engine was constructed to interpret topologies through parsing of lexical templates.
    These templates are technically implemented through the data format \myac{JSON}.
    This is a web-service friendly data format, which is commonly used to exchange data
    over web-based communication channels.
    This fact effectively makes the engine adaptable as a web-service end-point.
    \myac{JSON} is a human-readable language, which let end users create and edit
    templates manually with any text-based editor.
  \item[External library.]
    24 cloud providers are supported by the engine, through the advantage of 
    utilizing an external library.
    Such ``\emph{bridge}'' libraries connect the engine to cloud providers.
    The engine is designed so these ``\emph{bridge}'' libraries are abstracted
    behind a version of the facade pattern.
    The library focused upon in this thesis is \emph{jclouds},
    a library written in Java.
    This library is fluently added as a dependency in the Engine's Maven dependency.
  \item[Models@run.time.]
    \myac{M@RT} was utilized and combined with observer pattern to
    achieve asynchronous provisioning.
    As provisioning of one single node can take up to minutes to complete it was
    essential to implement asynchronous functionality.
    To achieve desired asynchronous behavior in CloudML,
    the engine exploited the advantage of combining observer pattern with Scala's
    built in actor model.
\end{description}
