\mychapter{conclusions}{Conclusions}

In this thesis four main parts have been presented.
First the background part introducing the domain of cloud computing and model-driven engineering.
Then the second part highlighting sets of technologies, frameworks, ideas and 
\myac{API}s which are currently used in the two domains.
Third the challenges with these solutions are stressed, as well as a set of \emph{requirements}
CloudML must fulfill to tackle these challenges.
Lastly, CloudML is presented, in three phases,
\begin{ii}
  \iitem vision,
  \iitem design and
  \iitem implementation.
\end{ii}

In the vision chapter the core idea of CloudML were introduced,
and even means to tackle \citereq{m@rt} were outlined through pure vision.
The design chapter stated how CloudML should be built up,
what the meta-model should look like,
what underlying technologies should be used.
All through a scenario where Alice performs provisioning.
In this chapter the means to tackle \citereq{m@rt} are reinforced through
the view of design.
The requirements of \citereq{foundation} and \citereq{software-reuse}
are addressed through what underlying technology to use and alternatives.
Lastly the implementation chapter outline how CloudML is implemented
as \emph{cloudml-engine}, and how this solution is built up.
Both \citereq{mda} and \citereq{lexical-template} are tackled in this chapter
by concretely choosing data format and syntax based on the design chapter.

\section{Results}
\begin{table}
    %\begin{tabular}{ | p{2cm} | p{2cm} | p{2.5cm} | p{2cm} | p{2cm} | p{2cm} |}
  \begin{tabular*}{\textwidth}{@{\extracolsep{\fill}}| l | l | l | l | l | l |}
    %\begin{tabular*}{\textwidth}{ | l | l | l | l | l | l |}
      \hline
        \textbf{State of the art} & 
        \textbf{\citereq{software-reuse}} & 
        \textbf{\citereq{foundation}} & 
        \textbf{\citereq{mda}} & 
        \textbf{\citereq{m@rt}} & 
        \textbf{\citereq{lexical-template}} \\
      \hline
     Amazon CloudFormation & No & Hard & No & No & No \\ \hline
     CA Applogic & Yes & Easy & Yes & N & No  \\ \hline
     Libcloud & No & Hard & No & Yes & No \\ \hline
     jclouds & No & Hard & No & Yes & No \\ \hline
     OPA & Yes & Hard & No & No & No \\ \hline
     Whirr & No & Hard & No & Yes & No \\ \hline
     Deltacloud & No & Hard & No & Yes& No  \\ \hline
     CloudML & Yes & Easy & Yes & Yes & No \\ \hline
  \end{tabular*}
  \caption{Analysis}
  \label{table:analysis}
\end{table}



The implementation is validated through an experiment where
it is physically executed against two providers, \myac{AWS} and Rackspace.
This experiment concludes the work put into CloudML at this point to be successful.

The requirements from \citechap{requirements} are compared against selected
technologies and frameworks from \citechap{state-of-the-art}.
These comparisons are expressed in \citetable{requirements-comparison}.

These are the ``\emph{key results}'' composed by this thesis:
\begin{enumerate}
  \item A meta-model was designed, with model-driven engineering in mind.
  \item An engine capable of provisioning nodes on a set of supported cloud providers
    was implemented.
  \item The engine was constructed to interpret topologies through parsing of lexical files.
  \item 24 cloud providers are supported by the engine, through the advantage of 
    utilizing an external library.
  \item \myac{M@RT} was utilized and combined with observer pattern to
    achieve asynchronous provisioning.
\end{enumerate}
