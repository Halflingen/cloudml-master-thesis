\mychapter{conclusions}{Conclusions}

In this thesis four main parts have been presented.
First the background part introducing the domain of cloud computing and model-driven engineering.
Then the second part highlighting sets of technologies, frameworks, ideas and 
\myac{API}s which are currently used in the two domains.
Third the challenges with these solutions are stressed, as well as a set of \emph{requirements}
CloudML must fulfill to tackle these challenges.
Lastly, CloudML is presented, in three phases,
\begin{ii}
  \iitem vision,
  \iitem design and
  \iitem implementation.
\end{ii}

In the vision chapter the core idea of CloudML were introduced,
and even means to tackle \citereq{m@rt} were outlined through pure vision.
The design chapter stated how CloudML should be built up,
what the meta-model should look like,
what underlying technologies should be used.
All through a scenario where Alice performs provisioning.
In this chapter the means to tackle \citereq{m@rt} are reinforced through
the view of design.
The requirements of \citereq{foundation} and \citereq{software-reuse}
are addressed through what underlying technology to use and alternatives.
Lastly the implementation chapter outline how CloudML is implemented
as \emph{cloudml-engine}, and how this solution is built up.
Both \citereq{mda} and \citereq{lexical-template} are tackled in this chapter
by concretely choosing data format and syntax based on the design chapter.

\paragraph{Results.}

The implementation is validated through an experiment where
it is physically executed against two providers, \myac{AWS} and Rackspace.
This experiment concludes the work put into CloudML at this point to be successful.

The requirements from \citechap{requirements} are compared against selected
technologies and frameworks from \citechap{state-of-the-art}.
These comparisons are expressed in \citetable{requirements-comparison}.

These are the \emph{``key results''} composed by this thesis:
\begin{description}
  \item[Meta-model.]
    A meta-model was designed, with model-driven engineering in mind.
    The meta-model was presented through a scenario where the user
    provisioned nodes based on two specific topologies.
    The first topology consisted of a single node, to exemplify the simplest topology.
    The second setup was based on three nodes, representing a common multi-tier
    topology for development purposes.
    Lastly the intention of a multi-cloud provision with the meta-model was described.
  \item[Engine.]
    An engine capable of provisioning nodes on a set of supported cloud providers
    was implemented.
    The engine took advantage of the automatic building system Maven,
    enhancing the support for internal and external modules.
    Through Maven the engine became increasingly adaptable for developers
    seeking to include or utilize CloudML.
    Internally the engine was split into four logical sub-modules.
    Scala was used as programming language for the engine,
    providing a \emph{state-of-the-art} set of features,
    \eg combination of object-oriented and functional programming.
  \item[Lexical templates.]
    The engine was constructed to interpret topologies through parsing of lexical templates.
    These templates were technically implemented through the data format \myac{JSON}.
    This is a web-service friendly data format, which is commonly used to exchange data
    over web-based communication channels.
    This fact effectively made the engine adaptable as a web-service end-point.
    \myac{JSON} is a human-readable language, which let end users create and edit
    templates manually with any text-based editor.
  \item[External library.]
    An external library was utilized, giving the engine support for $24$ providers,
    through the advantage of this library.
    Such \emph{``bridge''} libraries connected the engine to cloud providers.
    The engine was designed so these \emph{``bridge''} libraries were abstracted
    behind a version of the facade pattern.
    The library focused upon in this thesis was \emph{jclouds},
    a library written in Java.
    This library was fluently added as a dependency in the Engine's Maven dependency.
  \item[Models@run.time.]
    \myac{M@RT} was utilized and combined with observer pattern to
    achieve asynchronous provisioning.
    As provisioning of one single node can take up to minutes to complete it was
    essential to implement asynchronous functionality.
    To achieve desired asynchronous behavior in CloudML,
    the engine exploited the advantage of combining observer pattern with Scala's
    built in actor model.
\end{description}
