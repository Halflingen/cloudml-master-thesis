\mychapter{problem}{Challenges in the cloud}

As cloud computing is growing in popularity it is also growing in complexity.
More and more providers are entering the market and different types of solutions are made.
There are few physical restrictions on how a provider should let their users do provisioning,
and little limitations in technological solutions.  
The result can be a complex and struggling introduction to cloud computing for users,
and provisioning procedure can alternate between providers.

This chapter will outline research on which has been conducted by
physical provisioning of an example application.
First the scenario will be introduced, describing the example application
and different means of provisioning in form of topologies.
Then challenges identified from the research will be presented.

\section{Scenario}

The following scenario was chosen because of how much it resembles actual solutions
used in industry today.
It uses a featureless example application meant to fit into scenario topologies
without having too much complexity.
Challenges should not be affected from errors or problems with the example application.
The application will be provisioned to a defined set of providers with a defined set of different topologies.

\paragraph{BankManager.}

To recognize challenges when doing cloud provisioning an example application~\cite{BankManager} was utilized.
The application (from here known as \emph{BankManager}) is a prototypical bank manager system
which support creating users and bank accounts and moving money between bank accounts and users.
The application is based on a three-tier architecture with 
\begin{ii} 
  \iitem presentation tier with a web-based interface,
  \iitem logic tier with controllers and services and
  \iitem database tier with models and entities.
\end{ii}
Three or more tiers in a web application is a common solution, even more so for applications 
based on the \myac{MVC} architectural pattern.
The advantage with this architecture is that the lowest tier (database) can be physically
detached from the tiers above, the application can then be distributed between several nodes.
It is also possible to have more tiers, for instance by adding a \emph{service} 
layer to handle re-usable logic.
Having more tiers and distributing these over several nodes is an architecture often
found in \myac{SOA} solutions.

\paragraph{Topologies.}

\begin{figure}
  \tikzstyle{browser}=[rectangle, dashed, draw=black, rounded corners, fill=white!40, drop shadow,
  text centered, anchor=north, text=black, text width=3.5cm, top color=white, bottom color=black!10]

    \subfigure[Single node]{
      \begin{tikzpicture}[scale=0.7, transform shape]
        \node (Browser) [browser, text width=1.5cm] { \textbf{Browser} };
        \node (Node) [class, right=of Browser] { \textbf{Front-end And Back-end} };

        \draw[extend] (Browser.east) -- (Node.west);
      \end{tikzpicture}
      \label{fig:singlenode}
    }

    \subfigure[Two nodes]{
      \begin{tikzpicture}[scale=0.7, transform shape]
        \node (Browser) [browser, text width=1.5cm] { \textbf{Browser} };
        \node (Frontend) [class, right=of Browser] { \textbf{Front-end} };
        \node (Backend) [class, right=of Frontend] { \textbf{Back-end} };

        \draw[extend] (Browser.east) -- (Frontend.west);
        \draw[extend] (Frontend.east) -- (Backend.west);
      \end{tikzpicture}
      \label{fig:twonodes}
    }

    \subfigure[Three nodes]{
      \begin{tikzpicture}[scale=0.7, transform shape]
        \node (Browser) [browser, text width=1.5cm] { \textbf{Browser} };
        \node (LoadBalancer) [tdiamond, right=of Browser] { \textbf{Load balancer} };
        \node (Frontend1) [class, right=of LoadBalancer, yshift=1cm] { \textbf{Front-end} };
        \node (Frontend2) [class, below=of Frontend1] { \textbf{Front-end} };
        \node (Backend) [class, right=of Frontend1, yshift=-1cm] { \textbf{Back-end} };

        \draw[extend] (Browser.east) -- (LoadBalancer.west);
        \draw[extend] (LoadBalancer.east) -- (Frontend1.west);
        \draw[extend] (LoadBalancer.east) -- (Frontend2.west);
        \draw[extend] (Frontend1.east) -- (Backend.west);
        \draw[extend] (Frontend2.east) -- (Backend.west);
      \end{tikzpicture}
      \label{fig:threenodes}
    }

    \subfigure[Several front-ends]{
      \begin{tikzpicture}[scale=0.7, transform shape]
        \node (Browser) [browser, text width=1.5cm] { \textbf{Browser} };
        \node (LoadBalancer) [tdiamond, right=of Browser] { \textbf{Load balancer} };
        \node (Frontend1) [class, right=of LoadBalancer, yshift=1cm] { \textbf{Front-end} };
        \node (Frontend2) [class, below=of Frontend1, yshift=-1cm] { \textbf{Front-end} };
        \node (Backend) [class, right=of Frontend1, yshift=-1cm] { \textbf{Back-end} };

        \draw[extend] (Browser.east) -- (LoadBalancer.west);
        \draw[extend] (LoadBalancer.east) -- (Frontend1.west);
        \draw[extend] (LoadBalancer.east) -- (Frontend2.west);
        \draw[extend] (Frontend1.east) -- (Backend.west);
        \draw[extend] (Frontend2.east) -- (Backend.west);

        \draw[loosely dotted, line width=3pt] (Frontend1.south) -- (Frontend2.north);
      \end{tikzpicture}
      \label{fig:frontends}
    }

    \subfigure[Several front-ends and back-ends (slaves)]{
      \begin{tikzpicture}[scale=0.7, transform shape]
        \node (Browser) [browser, text width=1.5cm] { \textbf{Browser} };
        \node (LoadBalancer) [tdiamond, right=of Browser] { \textbf{Load balancer} };
        \node (Frontend1) [class, right=of LoadBalancer, yshift=1cm] { \textbf{Front-end} };
        \node (Frontend2) [class, below=of Frontend1, yshift=-1cm] { \textbf{Front-end} };
        \node (Backend) [class, right=of Frontend1, yshift=-1cm] { \textbf{Back-end master} };
        \node (Slave1) [class, right=of Backend, yshift=1cm] { \textbf{Slave} };
        \node (Slave2) [class, below=of Slave1, yshift=-1cm] { \textbf{Slave} };

        \draw[extend] (Browser.east) -- (LoadBalancer.west);
        \draw[extend] (LoadBalancer.east) -- (Frontend1.west);
        \draw[extend] (LoadBalancer.east) -- (Frontend2.west);
        \draw[extend] (Frontend1.east) -- (Backend.west);
        \draw[extend] (Frontend2.east) -- (Backend.west);
        \draw[extend] (Backend.east) -- (Slave1.west);
        \draw[extend] (Backend.east) -- (Slave2.west);

        \draw[loosely dotted, line width=3pt] (Frontend1.south) -- (Frontend2.north);
      \end{tikzpicture}
      \label{fig:frontendbackends}
    }

    \subfigure[Legend]{
      \begin{tikzpicture}[scale=0.7, transform shape]
        \node (Browser) [browser, label=below:Non-system interaction] { Browser };
        \node (Node) [class, right=of Browser, label=below:Provisioned instance] { Node };
        \node (LoadBalancer) [tdiamond, right=of Node, label=below:Load balancer as a service] { Load balancer};

        \node (AuxNode01) [right=of LoadBalancer] {};
        \node (AuxNode02) [right=of AuxNode01] {};
        \node (AuxNode03) [right=of AuxNode02] {};
        \node (AuxNode04) [right=of AuxNode03] {};

        \draw[extend] (AuxNode01) -- node[below] {Connection flow} (AuxNode02);
        \draw[loosely dotted, line width=3pt] (AuxNode03) -- node[below] {n-times} (AuxNode04);
      \end{tikzpicture}
    }

  \caption{Different architectural ways to provision nodes (topologies).}
  \label{fig:BankManager}
\end{figure}


Some examples of provisioning topologies are illustrated in \citefig{BankManager},
each example includes a \texttt{browser} to visualize application flow,
\texttt{front-end} visualizes executable logic and \texttt{back-end} represents database.
It is possible to have both \texttt{front-end} and \texttt{back-end} on the same node, 
as shown in \citefig{singlenode}.
When the topology have several \texttt{front-ends} a \texttt{load balancer} is used
to direct traffic between \texttt{browser} and \texttt{front-end}.
The \texttt{load balancer} could be a node like the rest, but in this cloud-based scenario
it is actually a cloud service, which is also why it is graphically different.
In \citefig{twonodes} \texttt{front-end} is separated from \texttt{back-end},
this introduces the flexibility of increasing computation power on the \texttt{front-end} node while spawning more
storage on the \texttt{back-end}.
For applications performing heavy computations it can be beneficial to distribute the workload between several
\texttt{front-end} nodes as seen in \citefig{threenodes}, the number of \texttt{front-ends} can be linearly increased
$n$ number of times as shown in \citefig{frontends}.
\emph{BankManager} is not designed to handle several \texttt{back-ends} because of \myac{RDBMS},
this can solved on a database level with master and slaves (\citefig{frontendbackends}).

\paragraph{Execution.}

The main goal of the scenario was to successfully deploy \emph{BankManager}
on a given set of providers with a given set of topologies.
And to achieve such deployment it was crucial to perform cloud provisioning.
The providers chosen were 
\begin{ii}
  \iitem \myac{AWS}~\cite{aws} and
  \iitem Rackspace~\cite{rackspace}.
\end{ii}
These are strong providers with a respectable amount of customers, as two of the leaders in cloud computing
[\todo{source}].
They also have different graphical interfaces, APIs and toolchains which makes them suitable
for a scenario researching multicloud challenges.

The topology chosen for this scenario was that of three nodes\citefig{threenodes}.
This topology is advance enough that it needs a \texttt{load balancer} in front of two
\texttt{front-end} nodes, and yet the simplest topology of the ones that benefits from a \texttt{load balancer}.
It is important to include most of the technologies and services that needs testing.

To perform the actual provisioning a set of primitive Bash-scripts were developed.
These scripts were designed to automate a full deployment on a two-step basis.
First step was to provision instances:
\begin{itemize}
  \item Authenticate against provider.
  \item Create instances.
  \item Manually write down IP addresses of created instances.
\end{itemize}
The second step was deployment:
\begin{itemize}
  \item Configure \emph{BankManager} to use one of provisioned instances IP address for database.
  \item Build \emph{BankManager} into a \myac{WAR}-file.
  \item Authenticate to instance using \myac{SSH}.
  \item Remotely execute commands to install required third party software such as Java and PostgreSQL.
  \item Remotely configure third party software.
  \item Inject \myac{WAR}-file into instances using \myac{SFTP}.
  \item Remotely start \emph{BankManager}.
\end{itemize}
The scripts were provider-specific so one set of scripts had to be made for each provider.
Rackspace had at that moment no command-line tools, so a \myac{REST} client had to be constructed.

\section{challenges}

From this research it became clear that there were multiple challenges to address
when deploying applications to cloud infrastructure.
This thesis is scoped to cloud provisioning, but the goal of this provisioning is to 
enable a successful deployment. 
It was therefore crucial to involve a full deployment in the scenario to discover
important challenges.

\paragraph{Complexity.} 

The first challenge encountered was to simply 
authenticate and communicate with the cloud. 
The two providers had different approaches, \myac{AWS}~\cite{aws} 
had command-line tools built from their Java APIs,
while Rackspace~\cite{rackspace} had no tools beside the API language bindings.
So for \myac{AWS} the Bash-scripts could do callouts to the command-line interface 
while for Rackspace the public \myac{REST} API had to be utilized.
This emphasized the inconsistencies between providers, 
and resulted in an additional tool being introduced to handle callouts.

As this emphasizes the complexity even further it also stresses engineering capabilities
of individuals executing the tasks to a higher technical level.

\paragraph{Multicloud.}

Once able to provision the correct amount of nodes with desired properties
on the first provider it became clear that mirroring the setup to the other provider 
was not as convenient as anticipated.
There were certain aspects of vendor lock-in, so each script was hand-crafted for specific providers.
The lock-in situations can in many cases have financial implications where for example
a finished application is locked to one provider and this provider increases tenant costs.
Or availability decreases and results in decrease of service uptime damaging revenue.

\paragraph{Reproducibility.}

The scripts provisioned nodes based on command-line arguments
and did not persist the designed topology in any way, this made topologies cumbersome to reproduce.

\paragraph{Shareable.}

Since the scripts did not remember a given setup it was impossible 
to share topologies ``as is'' between coworkers.
It is important that topologies can be shared because direct input from individuals
with different areas of competence can increase quality.

\paragraph{Robustness.}

There were several ways the scripts could fail and most errors were ignored.
Transactional behaviors were non-existent.

\paragraph{Run-time dependency.}

The scripts were developed to fulfill a complete deployment,
and to do this it proved important to temporally save run-time specific metadata.
This was crucial data needed to connect front-end nodes with the back-end node.


\section{Why is it important to solve the problems}

\begin{table}
    \begin{tabular}{ | p{2cm} | p{2cm} | p{2.5cm} | p{2cm} | p{2cm} | p{2cm} |}
      \hline
      \textbf{Solution} & \textbf{Learning curve} & \textbf{Business level viable} & \textbf{Model driven} & \textbf{Multicloud} \\
      \hline
     Amazon CloudFormation & No & Hard & No & No \\ \hline
     CA Applogic & Yes & Easy & Yes & N  \\ \hline
     Libcloud & No & Hard & No & Yes \\ \hline
     jclouds & No & Hard & No & Yes \\ \hline
     OPA & Yes & Hard & No & No \\ \hline
     Whirr & No & Hard & No & Yes \\ \hline
     Deltacloud & No & Hard & No & Yes \\ \hline
     CloudML & Yes & Easy & Yes & Yes \\ \hline
  \end{tabular}
  \caption{Analysis}
  \label{table:analysis}
\end{table}



\todo{
  \begin{itemize}
    \item Cloud domain is state of the art
    \item model driven approach with benefits (no special tooling)
    \item Easier for businesses (especially SMBs) to reach out to Cloud
    \item Easier for larger more time-constraint businesses to try out the cloud
    \item Opening the eyes of big providers for a larger cross-cloud language
  \end{itemize}
}
