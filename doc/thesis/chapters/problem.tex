\chapter{Problem definition with examples}

\todo{
  \begin{itemize}
    \item Outline the problem \\
      \begin{itemize}
        \item Information dependency at runtime
        \item Technical competence/level expectations
        \item Reproducibility
        \item Robustness
        \item Complexity
        \item Shareable
      \end{itemize}
    \item Why is it important to solve the problems
      \begin{itemize}
        \item Cloud domain is state of the art
        \item model driven approach with benefits (no special tooling)
        \item Easier for businesses (especially SMBs) to reach out to Cloud
        \item Easier for larger more time-constraint businesses to try out the cloud
        \item Opening the eyes of big providers for a larger cross-cloud language
      \end{itemize}
  \end{itemize}
}

There are many cloud providers on the global market today. These providers support many layers of cloud, 
such as PaaS (Platform as a Service) and IaaS (Infrastructure as a Service). 
This vast amount of providers and new technologies and services can be overwhelming for many companies and small and medium businesses. 
There are no practical introductions to possibilities and limitations to cloud computing, or the differences between different providers and services. 
Each provider has some kind of management console, usually in form of a web interface and API. 
But model driven approaches are inadequate in many of these environments. 
UML diagrams such as deployment diagram and component diagram are used in legacy systems to describe system architectures, 
but this advantage has yet to hit the mainstream of cloud computing management. 
It is also difficult to have co-operational interaction on a business level without using the advantage of graphical models.
The knowledge needed to handle one provider might differ to another, so a multicloud approach might be very resource-heavy on competence in companies. 
The types of deployment resources are different between the providers, even how to gain access to and handle running instances might be very different. 
Some larger cloud management application developers are not even providers themselves, but offer tooling for private cloud solutions.
Some of these providers have implemented different types of web based applications that let end users manage their cloud instances. 
The main problem with this is that there are no standards defining a cloud instance or links between instances and other services a provider offer.
If a provider does not offer any management interface and want to implement this as a new feature for customers, 
a standard format to set the foundation would help them achieve a better product for their end users.
These are some of the problems with cloud hosting today, and that CloudML will be designed to solve.
