\mychapter{related-works}{Related Works}

There already exists scientific research projects and technologies
which have similarities to CloudML both in idea and implementation. 
First scientific research projects will be presented with their solutions, 
then pure technological approaches will be introduced.

One project that bears relations to CloudML is mOSAIC~\cite{portable:petcu12} which
aims at not only provisioning in the cloud, but deployment as well.
They focus on abstractions for application developers and state they can easily enable users to
\emph{``obtain the desired application characteristics (like
scalability, fault-tolerance, QoS, \etc.)''}~\cite{architecturing:petcu11}.
The strongest similarities to CloudML are 
\begin{ii}\iitem multicloud with their API~\cite{architecturing:petcu11},
\iitem metadata dependencies since they support full deployment and
\iitem robustness through fault-tolerance.
What mOSAIC is lacking compared to CloudML is model-based approach including \emph{M@RT}.
Reservoir~\cite{reservoir:rochweger09} is another project that also aim at
\iitem multicloud. The other goals of this project is to leverage 
scalability in single providers and support built-in \emph{Business Service Management}~(BSM),
important topics but not directly related to the goals of CloudML.
CloudML stands out from Reservoir in the same way as mOSAIC.
Vega framework~\cite{simplifying:chieu10} is a deployment framework aiming 
at full cloud deployments of multi-tier topologies, they also follow a \iitem model-based 
approach\end{ii}. The main difference between CloudML and Vega are support of multicloud provisioning.

There are also distinct technologies that bear similarities to CloudML.
None of AWS CloudFormation and CA Applogic are \begin{ii}\iitem model-driven.
Others are plain APIs supporting \iitem multicloud such as libcloud, jclouds and DeltaCloud.
The last group are projects that aim specifically at deployment, 
making \emph{Infrastructure-as-a-Service}~(IaaS) work as \emph{Platform-as-a-Service}~(PaaS)
like AWS Beanstalk and SimpleCloud.\end{ii}
The downside about the technical projects are their inability to solve all of the challenges
that CloudML aims to address, but since these projects solve specific
challenges it is appropriate to utilize them.
Cloudml-engine leverages on jclouds in its implementation to support multicloud provisioning,
and future versions can utilize it for full deployments.
