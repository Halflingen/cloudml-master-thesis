\mychapter{challenges}{Challenges in the cloud}

As cloud computing is growing in popularity it is also growing in complexity.
More and more providers are entering the market and different types of solutions are made.
There are few physical restrictions on how a provider should let their users do provisioning,
and little limitations in technological solutions.
The result can be a complex and struggling introduction to cloud computing for users,
and provisioning procedure can alternate between providers.

This chapter will outline research on which has been conducted by
physical provisioning of an example application.
First the scenario will be introduced, describing the example application
and different means of provisioning in form of topologies.
Then the challenges identified from the research will be presented.

\section{Scenario}

The following scenario is chosen because of how much it resembles actual solutions
used in industry today.
It uses a featureless example application meant to fit into scenario topologies
without having too much complexity.
Challenges should not be affected from errors or problems with the example application.
The application will be provisioned to a defined set of providers with a defined set of different topologies.

\paragraph{BankManager.}

To recognize challenges when doing cloud provisioning an example application~\cite{BankManager} is utilized.
The application (from here known as \emph{BankManager}) is a prototypical bank manager system
which support creating users and bank accounts and moving money between bank accounts and users.
The application is based on a three-tier architecture with 
\begin{ii} 
  \iitem presentation tier with a web-based interface,
  \iitem logic tier with controllers and services and
  \iitem database tier with models and entities. \end{ii}Three 
or more tiers in a web application is a common solution, especially for applications 
based on the \myac{MVC} architectural pattern.
The advantage with this architecture is that the lowest tier (database) can be physically
detached from the tiers above, the application can then be distributed between several nodes.
It is also possible to have more tiers, for instance by adding a \emph{service} 
layer to handle re-usable logic.
Having more tiers and distributing these over several nodes is an architecture often
found in \myac{SOA} solutions.

\paragraph{Topologies.}
\begin{figure}[tb]
  \tikzstyle{browser}=[rectangle, dashed, draw=black, rounded corners, fill=white!40, drop shadow,
  text centered, anchor=north, text=black, text width=3.5cm, top color=white, bottom color=black!10]

  \begin{center}
    \subfigure[Single node]{
      \begin{tikzpicture}[scale=0.7, transform shape]
        \node (Browser) [browser, text width=1.5cm] { \textbf{Browser} };
        \node (Node) [class, right=of Browser] { \textbf{Front-end And Back-end} };

        \draw[extend] (Browser.east) -- (Node.west);
      \end{tikzpicture}
      \label{fig:singlenode}
    }

    \subfigure[Two nodes]{
      \begin{tikzpicture}[scale=0.7, transform shape]
        \node (Browser) [browser, text width=1.5cm] { \textbf{Browser} };
        \node (Frontend) [class, right=of Browser] { \textbf{Front-end} };
        \node (Backend) [class, right=of Frontend] { \textbf{Back-end} };

        \draw[extend] (Browser.east) -- (Frontend.west);
        \draw[extend] (Frontend.east) -- (Backend.west);
      \end{tikzpicture}
      \label{fig:twonodes}
    }

    \subfigure[Three nodes]{
      \begin{tikzpicture}[scale=0.7, transform shape]
        \node (Browser) [browser, text width=1.5cm] { \textbf{Browser} };
        \node (LoadBalancer) [tdiamond, right=of Browser] { \textbf{Load balancer} };
        \node (Frontend1) [class, right=of LoadBalancer, yshift=1cm] { \textbf{Front-end} };
        \node (Frontend2) [class, below=of Frontend1] { \textbf{Front-end} };
        \node (Backend) [class, right=of Frontend1, yshift=-1cm] { \textbf{Back-end} };

        \draw[extend] (Browser.east) -- (LoadBalancer.west);
        \draw[extend] (LoadBalancer.east) -- (Frontend1.west);
        \draw[extend] (LoadBalancer.east) -- (Frontend2.west);
        \draw[extend] (Frontend1.east) -- (Backend.west);
        \draw[extend] (Frontend2.east) -- (Backend.west);
      \end{tikzpicture}
      \label{fig:threenodes}
    }

    \subfigure[Several front-ends]{
      \begin{tikzpicture}[scale=0.7, transform shape]
        \node (Browser) [browser, text width=1.5cm] { \textbf{Browser} };
        \node (LoadBalancer) [tdiamond, right=of Browser] { \textbf{Load balancer} };
        \topologyntimes{right=of LoadBalancer}{Frontend}{Front-end}
        \node (Backend) [class, right=of Frontend-1, yshift=-1cm] { \textbf{Back-end} };

        \draw[extend] (Browser.east) -- (LoadBalancer.west);
        \draw[extend] (LoadBalancer.east) -- (Frontend-1.west);
        \draw[extend] (LoadBalancer.east) -- (Frontend-2.west);
        \draw[extend] (LoadBalancer.east) -- (Frontend-n.west);
        \draw[extend] (Frontend-1.east) -- (Backend.west);
        \draw[extend] (Frontend-2.east) -- (Backend.west);
        \draw[extend] (Frontend-n.east) -- (Backend.west);
      \end{tikzpicture}
      \label{fig:frontends}
    }

    \subfigure[Several front-ends and back-ends (slaves)]{
      \begin{tikzpicture}[scale=0.6, transform shape]
        \node (Browser) [browser, text width=1.5cm] { \textbf{Browser} };
        \node (LoadBalancer) [tdiamond, right=of Browser] { \textbf{Load balancer} };
        \topologyntimes{right=of LoadBalancer}{Frontend}{Front-end}
        \node (Backend) [class, right=of Frontend-1, yshift=-1cm] { \textbf{Back-end master} };
        \topologyntimes{right=of Backend}{Slave}{Slave}

        \draw[extend] (Browser.east) -- (LoadBalancer.west);
        \draw[extend] (LoadBalancer.east) -- (Frontend-1.west);
        \draw[extend] (LoadBalancer.east) -- (Frontend-2.west);
        \draw[extend] (LoadBalancer.east) -- (Frontend-n.west);
        \draw[extend] (Frontend-1.east) -- (Backend.west);
        \draw[extend] (Frontend-2.east) -- (Backend.west);
        \draw[extend] (Frontend-n.east) -- (Backend.west);
        \draw[extend] (Backend.east) -- (Slave-1.west);
        \draw[extend] (Backend.east) -- (Slave-2.west);
        \draw[extend] (Backend.east) -- (Slave-n.west);
      \end{tikzpicture}
      \label{fig:frontendbackends}
    }

    \subfigure[Legend]{
      \begin{tikzpicture}[scale=0.7, transform shape]
        \node (Browser) [browser, label=below:Non-system interaction] { Browser };
        \node (Node) [class, right=of Browser, label=below:Provisioned instance] { Node };
        \node (LoadBalancer) [tdiamond, right=of Node, label=below:Load balancer as a service] { Load balancer};

        \node (AuxNode01) [right=of LoadBalancer] {};
        \node (AuxNode02) [right=of AuxNode01] {};
        \node (AuxNode03) [right=of AuxNode02] {};
        \node (AuxNode04) [right=of AuxNode03] {};

        \draw[extend] (AuxNode01) -- node[below] {Connection flow} (AuxNode02);
        \draw[loosely dotted, line width=3pt] (AuxNode03) -- node[below] {n-times} (AuxNode04);
      \end{tikzpicture}
    }
  \end{center}

  \caption{Different architectural ways to provision nodes (topologies).}
  \label{fig:topologies}
\end{figure}


Some examples of provisioning topologies are illustrated in \citefig{topologies}.
Each example includes a \texttt{browser} to visualize application flow,
\texttt{front-end} visualizes executable logic and \texttt{back-end} represents database.
It is possible to have both \texttt{front-end} and \texttt{back-end} on the same node, 
as shown in \citefig{singlenode}.
When the topology have several \texttt{front-ends} a \texttt{load balancer} is used
to direct traffic between \texttt{browser} and \texttt{front-end}.
The \texttt{load balancer} could be a node like the rest, but in this cloud-based scenario
it is actually a cloud service, which is also why it is graphically different.
In \citefig{twonodes} \texttt{front-end} is separated from \texttt{back-end},
this introduces the flexibility of increasing computation power on the \texttt{front-end} node while spawning more
storage on the \texttt{back-end}.
For applications performing heavy computations it can be beneficial to distribute the workload between several
\texttt{front-end} nodes as seen in \citefig{threenodes}, the number of \texttt{front-ends} can be linearly increased
$n$ number of times as shown in \citefig{frontends}.
\emph{BankManager} is not designed to handle several \texttt{back-ends} because of \myac{RDBMS},
this can be solved at the database level with master and slaves (\citefig{frontendbackends}).

\paragraph{Execution.}

The main goal of the scenario is to successfully deploy \emph{BankManager}
on a given set of providers with a given set of topologies.
And to achieve such deployment it is crucial to perform cloud provisioning.
The providers chosen are
\begin{ii}
  \iitem \myac{AWS}~\cite{aws} and
  \iitem Rackspace~\cite{rackspace}.
\end{ii}
These are strong providers with a respectable amount of customers, as two of the leaders in cloud computing.
They also have different graphical interfaces, APIs and toolchains which makes them suitable
for a scenario addressing multicloud challenges.

The topology chosen for this scenario have three nodes,~\citefig{threenodes}.
This topology is advanced enough that it needs a \texttt{load balancer} in front of two
\texttt{front-end} nodes, and yet the simplest topology of the ones that benefits from a \texttt{load balancer}.
It is important to include most of the technologies and services that needs testing.

To perform the actual provisioning a set of primitive Bash-scripts are developed.
These scripts are designed to automate a full deployment on a two-step basis.
First step is to provision instances:
\begin{itemize}
  \item Authenticate against provider.
  \item Create instances.
  \item Manually write down IP addresses of created instances.
\end{itemize}
The second step is deployment:
\begin{itemize}
  \item Configure \emph{BankManager} to use one of provisioned instances IP address for database.
  \item Build \emph{BankManager} into a \myac{WAR}-file.
  \item Authenticate to instance using \myac{SSH}.
  \item Remotely execute commands to install required third party software such as Java and PostgreSQL.
  \item Remotely configure third party software.
  \item Inject \myac{WAR}-file into instances using \myac{SFTP}.
  \item Remotely start \emph{BankManager}.
\end{itemize}
The scripts are provider-specific so one set of scripts had to be made for each provider.
Rackspace had at that moment no command-line tools, so a \myac{REST} client had to be constructed.

\section{Challenges}

From this example it became clear that there were multiple challenges to address
when deploying applications to cloud infrastructure.
This thesis is scoped to cloud provisioning, but the goal of this provisioning step is to 
enable a successful deployment~(see \citechap{perspectives}).
It is therefore crucial to involve a full deployment in the scenario to discover
important challenges.

\paragraph{Complexity.} 

The first challenge encountered is to simply 
authenticate and communicate with the cloud. 
The two providers had different approaches, \myac{AWS}~\cite{aws} 
had command-line tools built from their Java APIs,
while Rackspace~\cite{rackspace} had no tools beside the \myac{API} language bindings.
So for \myac{AWS} the Bash-scripts could do callouts to the command-line interface 
while for Rackspace the public \myac{REST} \myac{API} had to be utilized.
This emphasized the inconsistencies between providers, 
and resulted in an additional tool being introduced to handle requests.

As this emphasizes the complexity even further it also stresses engineering 
capabilities of individuals.
It would be difficult for non-technical participants to fully understand and give comments
or feedback on the topology chosen since important information got hidden behind
complex commands.

\paragraph{Feedback on failure.}
Debugging the scripts is also a challenging task, since they fit together by
sequential calls and printed information based on Linux and Bash commands such as 
\emph{grep} and \emph{echo}.
Error messages from both command-line and \myac{REST} interfaces are essentially muted away.
If one specific script should fail it would be difficult to know 
\begin{ii}
  \iitem which script failed, 
  \iitem at what step it was failing and 
  \iitem what was the cause of failure
\end{ii}.

\paragraph{Multicloud.}

Once able to provision the correct amount of nodes with desired properties
on the first provider it became clear that mirroring the setup to the other provider 
is not as convenient as anticipated.
There are certain aspects of vendor lock-in, so each script is hand-crafted for specific providers.
The most noticeable differences would be
\begin{ii}
  \iitem different ways of defining instance sizes,
  \iitem different versions, distributions or types of operating systems (\emph{images}),
  \iitem different way of connection to provisioned instances
\end{ii}.
The lock-in situations can in many cases have financial implications where for example
a finished application is locked to one provider and this provider increases tenant costs.
Or availability decreases and results in decrease of service uptime damaging revenue.

\paragraph{Reproducibility.}

The scripts provisioned nodes based on command-line arguments
and did not persist the designed topology in any way.
This made topologies cumbersome to reproduce.
If the topology could be persisted in any way, for example serialized files,
it would be possible to reuse these files at a later time.
The persisted topologies could also be reused on other clouds making a 
similar setup at another cloud provider, or even distribute the setup
between providers.

\paragraph{Shareable.}

Since the scripts did not remember a given setup it is impossible 
to share topologies ``as is'' between coworkers.
It is important that topologies can be shared because direct input from individuals
with different areas of competence can increase quality.
If the topology could be serialized into files these files could also be interpreted
and loaded into different tools to help visualizing and editing.

\paragraph{Robustness.}

There are several ways the scripts could fail and most errors are ignored.
They are made to search for specific lines in strings returned by the APIs,
if these strings are non-existent the scripts would just continue regardless
of complete dependency to information within the strings.
A preferable solution to this could be transactional behavior with rollback functionality
in case an error should occur, or simply stop the propagation
and throw exceptions that can be handled on a higher level.

\paragraph{Metadata dependency.}

The scripts were developed to fulfill a complete deployment,
including 
\begin{ii}
  \iitem provisioning instances, 
  \iitem install third party software on instances,
  \iitem configure instances and software,
  \iitem configure and upload \myac{WAR}-file and
  \iitem deploy and start the application from the \myac{WAR}-file
\end{ii}.
In this thesis the focus is aimed at provisioning, but it proved important to temporally 
save run-time specific metadata to successfully deploy the application.
In the \emph{BankManager} example the crucial metadata is information needed to connect 
front-end nodes with the back-end node, but other deployments is likely to need the same 
or different metadata for other tasks.
This metadata is collected in step \iii{1}, and used in step \iii{3} and step \iii{4}.

\section{Summary}

There are many cloud providers on the global market today. 
These providers support many layers of cloud, such as \myac{PaaS} and \myac{IaaS}.
This vast amount of providers and new technologies and services can be overwhelming 
for many companies and small and medium businesses. 
There are no practical introductions to possibilities and limitations to cloud computing, 
or the differences between different providers and services. 
Each provider has some kind of management console, usually in form of a web interface and API. 
But model driven approaches are inadequate in many of these environments. 
\myac{UML} diagrams such as deployment diagram and component diagram are used in legacy systems 
to describe system architectures, 
but this advantage has yet to hit the mainstream of cloud computing management. 
It is also difficult to have co-operational interaction on a business level without 
using the advantage of graphical models.
The knowledge needed to handle one provider might differ to another, 
so a multicloud approach might be very resource-heavy on competence in companies. 
The types of deployment resources are different between the providers, 
even how to gain access to and handle running instances might be very different. 
Some larger cloud management application developers are not even providers themselves, 
but offer tooling for private cloud solutions.
Some of these providers have implemented different types of web based applications 
that let end users manage their cloud instances. 
The main problem with this is that there are no standards defining a 
cloud instance or links between instances and other services a provider offer.
If a provider does not offer any management interface and want to implement this as a new feature for customers, 
a standard format to set the foundation would help them achieve a better product for their end users.
