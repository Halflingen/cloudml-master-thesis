\mychapter{background}{Cloud computing and Model-Driven Artchitecture}
\begin{table}
  \begin{tabular}{ | l | l | l | }
    \hline
    \textbf{Provider} & \textbf{Service} & \textbf{Service layer} \\ \hline
    AWS & \emph{Elasic Compute Cloud}~(EC2) & IaaS \\ \hline
    AWS & Elastic Beanstalk & PaaS \\ \hline
    Google & \emph{Google App Engine}~(GAE) & PaaS \\ \hline
    Microsift & Azzure & PaaS and IaaS \\ \hline
    Heroku & Different services & PaaS \\ \hline
    Nodejitsu & Node.js & PaaS \\ \hline
    Rackspace & CloudServers & IaaS \\ \hline
  \end{tabular}
  \caption{Providers available services}
  \label{table:providerservices}
\end{table}


\begin{figure}
  \begin{center}
    \begin{tikzpicture}[scale=1.2, transform shape]
      \node[box] (IaaS) {IaaS};
      \node[box, above=of IaaS, yshift=-0.9cm] (PaaS) {PaaS};
      \node[box, above=of PaaS, yshift=-0.9cm] (SaaS) {SaaS};
    \end{tikzpicture}
  \end{center}
  \caption{Cloud architecture service models}
  \label{fig:cloud-models}
\end{figure}


Text here\ldots

\section{Cloud computing}

Cloud computing is gaining popularity and more companies are starting 
to explore the possibilities as well as the limitation to the cloud.
There are three main architectural service models in cloud computing\cite{nist:mell11}
namely \emph{Infrastructure-as-a-Service}~(IaaS), \emph{Platform-as-a-Service}~(PaaS)
and \emph{Software-as-a-Service}~(SaaS).
IaaS is on the lowest vertical integration level closest to physical hardware and SaaS on the highest
level as runnable applications.
Stanoevska-Slabeva~\cite{introduction:wozniak10} emphasizes that
\emph{''infrastructure had been available as a service for quite some time``} and this 
\emph{''has been referred to as utility computing``}, such as Sun Grid Compute Utility.

\paragraph{IaaS.}
The main providers are Google, Amazon with \emph{Amazon Web Service}~(AWS)~\cite{aws} and Microsoft.
A non-exhaustive list of common providers are visualized in \citetable{providerservices}.
The \emph{National Institute of Standards and Technology}~(NIST) is one of 
the leaders in cloud computing standardization.
The NIST Definition of Cloud Computing~\cite{nist:mell11} define IaaS as
\epigraph{The capability provided to the consumer is to provision 
  processing, storage, networks, and other fundamental computing resources where the 
  consumer is able to deploy and run arbitrary software, which can include operating 
  systems and applications.
}{\todo{NIST, 2011}}
These are capabilities found in cloud provider services, 
such as AWS \emph{Elastic Compute Cloud}~(EC2) and Rackspace CloudServers.
NIST continue to state that 
\epigraph{The consumer does not manage or control the underlying cloud 
  infrastructure but has control over operating systems, storage, deployed applications, and 
  possibly limited control of select networking components (\eg, host firewalls).
}{\todo{NIST, 2011}}
\paragraph{PaaS.}
The PaaS model is defined as an capability consumers use to deploy onto cloud infrastructure.
Deploying application that providers fully or partially support. For this kind of deployment
consumers do not have to manage or control underlying infrastructure capabilities,
and in some cases not even configuration.
Examples of PaaS providers are Google with \emph{Google App Engine}~(GAE) and
the company Heroku with their service with the same name.
Multiple PaaS providers utilize EC2 as underlying infrastructure, examples of such
providers are Heroku Nodester and Nodejitsu, this is a tendency with increasing popularity.
\paragraph{SaaS.}
The core purpose is to provide whole web applications as services, in many cases end products.
Google products such as gmail, Google Apps and  Google Calendar are examples of 
SaaS applications.

Some of the most essential characteristics of cloud computing~\cite{nist:mell11} are:
\begin{itemize}
  \item \emph{On-demand self-service}: Consumers can do provisioning without any human interaction
  \item \emph{Broad network access}: Capabilities available over standard network mechanisms
  \item \emph{Resource pooling}: Physical and virtual resources are dynamically assigned
    and reassigned according to consumer demand
  \item \emph{Rapid elasticity}: Automatic capability scaling
  \item \emph{Measured service}: Monitoring and control of resource usages
\end{itemize}

There are four different deployment models according to The 
NIST Definition of Cloud Computing~\cite{nist:mell11}:
\begin{itemize}
  \item \emph{Private cloud}: Similar to classical infrastructures where hardware and
    drifting is owned and controlled by organizations themselves.
  \item \emph{Community cloud}: When several organizations share the same aspects of
    a private cloud (such as security requirements, policies, and compliance considerations),
    and therefore share infrastructure.
  \item \emph{Public cloud}: Infrastructure is open to the public.
    Cloud providers own the hardware and rent out IaaS and PaaS solutions to users.
    Examples of such providers are Amazon with AWS and Google with GAE.
  \item \emph{Hybrid cloud}: Combining private cloud or community cloud with public cloud.
    One benefit is to distinguish data from logic for purposes such as security issues,
    by storing sensitive information in a private cloud while computing with public cloud.
\end{itemize}

Beside these models defined by NIST there is another arising model known as 
\emph{virtual private cloud}, which is similar to \emph{public cloud} 
but with some security implications such as sandboxed network.

\section{Model-Driven Architecture approach}

By combining the world of cloud computing with the one of modeling 
it is possible to achieve benefits such as improved communication when designing 
a system and better understanding of the system itself.
This statement is emphasized by Booch (with co-authors) in one of his studies:
\epigraph{
  ``Modeling is a central
  part of all the activities that lead up to the deployment of good
  software. We build models to communicate the desired structure and
  behavior of our system. We build models to visualize and control the
  system's architecture. We build models to better understand the
  system we are building, often exposing opportunities for
  simplification and reuse. We build models to manage risk.''
}{\todo{Booch, 2005}}
When it comes to cloud computing these definitions are even more important
because of financial aspects since provisioned nodes instantly draw credit.
The definition of ``modeling'' can be assessed from the previous epigraph, but it is 
also important to choose correct models for the task.
Stanoevska-Slabeva emphasizes in one of her studies that grid computing
``\emph{is the starting point and basis for Cloud Computing.}''~\cite{introduction:wozniak10}.
As grid computing bear similarities towards cloud computing in terms of vitalization and utility computing
it is possible to use the same UML diagrams for IaaS as previously used in grid computing.
The importance of this re-usability of models is based on the origination of grid computing, \emph{eScience},
and the popularity of modeling in this research area.
The importance of choosing correct models is emphasized by Booch~\cite{unified:booch05}:
\epigraph{
  \begin{ii}\iitem The choice
  of what models to create has a profound influence on how a problem
  is attacked and how a solution is shaped. \iitem Every model may be
  expressed at different levels of precision. \iitem The best models
  are connected to reality. \iitem No single model is
  sufficient. Every nontrivial system is best approached through a
  small set of nearly independent models.\end{ii}
}{\todo{Booch, 2005}}
These definition precepts state that several models (precept \iii{4}) on different levels (precept \iii{2}) 
of precision should be used to model the same system.
From this it is concludable that several diagrams can be used to describe one or several cloud computing perspectives.
Nor are there any restraints to only use UML diagrams or even diagrams at all.
As an example AWS CloudFormation implements a lexical model of their \emph{cloud services},
while CA AppLogic has a visual and more UML component-based diagram of their capabilities.
When working with \emph{Model-Driven Architecture}~(MDA) it is common to first create a
\emph{Computation Independent Model}~(CIM), then a \emph{Platform-Independent Model}~(PIM) and
lastly a \emph{Platform-Specific Model}~(PSM). There are other models and steps in between these,
but they render the essentials.
There are five different life cycles as explained by Singh~\cite{model-driven:singh09}:
\begin{enumerate}
  \item Create a CIM capturing requirements.
  \item Develop a PIM.
  \item Convert the PIM into PSM.
  \item Generate code form PSM.
  \item Deploy.
\end{enumerate}
It is important to know these aspects as it later in this paper will be explained what cycles are
specific to CloudML, what is scoped out and what is envisioned for future versions of CloudML.
