\chapter{Background}
\begin{table}
  \begin{tabular}{ | l | l | l | }
    \hline
    \textbf{Provider} & \textbf{Service} & \textbf{Service layer} \\ \hline
    AWS & \emph{Elasic Compute Cloud}~(EC2) & IaaS \\ \hline
    AWS & Elastic Beanstalk & PaaS \\ \hline
    Google & \emph{Google App Engine}~(GAE) & PaaS \\ \hline
    Microsift & Azzure & PaaS and IaaS \\ \hline
    Heroku & Different services & PaaS \\ \hline
    Nodejitsu & Node.js & PaaS \\ \hline
    Rackspace & CloudServers & IaaS \\ \hline
  \end{tabular}
  \caption{Providers available services}
  \label{table:providerservices}
\end{table}



\note{
  Explain some of the topics in my thesis. \\
  Here it is possible to introduce case study (BankManager) to ease writing
}

\todo {
  \begin{itemize}
    \item What is model-based engineering and benefits. \\
        Core concepts
  \end{itemize}
}

Cloud computing is gaining popularity and more companies are starting 
to explore the possibilities as well as the limitation to the cloud.
There are three main architectural service models in cloud computing\cite{nist:mell11}


namely \emph{Infrastructure-as-a-Service}~(IaaS), \emph{Platform-as-a-Service}~(PaaS)
and \emph{Software-as-a-Service}~(SaaS).
IaaS is on the lowest vertical integration level closest to physical hardware and SaaS on the highest
level as runnable applications.
\begin{itemize}
  \item \emph{IaaS}: 
    The main providers are Google, Amazon with \emph{Amazon Web Service}~(AWS)~\cite{aws} and Microsoft.
    Some of providers are visualized in \citetable{providerservices}.
    The NIST Definition of Cloud Computing~\cite{nist:mell11} define IaaS as
    \emph{``The capability provided to the consumer is to provision 
    processing, storage, networks, and other fundamental computing resources where the 
    consumer is able to deploy and run arbitrary software, which can include operating 
    systems and applications.''}
    And continue to state that \emph{``The consumer does not manage or control the underlying cloud 
    infrastructure but has control over operating systems, storage, deployed applications, and 
    possibly limited control of select networking components (\eg, host firewalls).``}
    These are capabilities found in cloud provider services, 
    such as AWS \emph{Elastic Compute Cloud}~(EC2) and Rackspace CloudServers.
    According to Katarina Stanoevska-Slabeva~\cite{introduction:wozniak10} have 
    \emph{''infrastructure had been available as a service for quite some time``} and this 
    \emph{''has been referred to as utility computing``}, such as Sun Grid Compute Utility.
    IaaS is the most relevant cloud architecture layer for this paper, with focus
    on cloud provisioning.
  \item \emph{PaaS}:
    The PaaS model is defined as an capability consumers use to deploy onto cloud infrastructure.
    Deploying application that providers fully or partially support. For this kind of deployment
    consumers do not have to manage or control underlying infrastructure capabilities,
    and in some cases not even configuration.
    Examples of PaaS providers are Google with \emph{Google App Engine}~(GAE) and
    the company Heroku with their service with the same name.
    Multiple PaaS providers utilize EC2 as underlying infrastructure, examples of such
    providers are Heroku Nodester and Nodejitsu, this is a tendency with increasing popularity.
  \item \emph{SaaS}:
    SaaS has less relevance to this paper, nevertheless the core purpose
    is to provide whole web applications as services, in many cases end products.
    Google products such as gmail, Google Apps and  Google Calendar are examples of 
    SaaS applications.
\end{itemize}

Some of the most essential characteristics of cloud computing~\cite{nist:mell11} are:
\begin{itemize}
  \item \emph{On-demand self-service}: Consumers can do provisioning without any human interaction
  \item \emph{Broad network access}: Capabilities available over standard network mechanisms
  \item \emph{Resource pooling}: Physical and virtual resources are dynamically assigned
    and reassigned according to consumer demand
  \item \emph{Rapid elasticity}: Automatic capability scaling
  \item \emph{Measured service}: Monitoring and control of resource usages
\end{itemize}

There are four different deployment models according to The 
NIST Definition of Cloud Computing~\cite{nist:mell11}:
\begin{itemize}
  \item \emph{Private cloud}: Similar to classical infrastructures where hardware and
    drifting is owned and controlled by organizations themselves.
  \item \emph{Community cloud}: When several organizations share the same aspects of
    a private cloud (such as security requirements, policies, and compliance considerations),
    and therefore share infrastructure.
  \item \emph{Public cloud}: Infrastructure is open to the public.
    Cloud providers own the hardware and rent out IaaS and PaaS solutions to users.
    Examples of such providers are Amazon with AWS and Google with GAE.
  \item \emph{Hybrid cloud}: Combining private cloud or community cloud with public cloud.
    One benefit is to distinguish data from logic for purposes such as security issues,
    by storing sensitive information in a private cloud while computing with public cloud.
\end{itemize}

Of these deployment models \emph{public cloud} is the most relevant model for this paper
since the main purpose is to provision on public providers such as Amazon and Rackspace.
Beside these models defined by NIST there is another arising model known as 
\emph{virtual private cloud}, which is similar to \emph{public cloud} 
but with some security implications such as sandboxed network.

