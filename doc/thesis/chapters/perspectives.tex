\mychapter{perspectives}{Perspectives}

In this chapter the future of CloudML is presented.
It faces two main topics.
The first about short term improvements to the implementation, both in features, enhancements and refactoring.
The second stresses the future of CloudML in terms of long term enhancements,
then not just for the implementation but in core design.

\section{Short term}

\paragraph{Load balancer.}
\begin{figure}[tb]
  \begin{center}
    \begin{minted}[mathescape,
                   linenos,
                   numbersep=5pt,
                   gobble=2,
                   frame=lines,
                   framesep=2mm]{json}
{
  "name": "test",
  "loadBalancer": {
    "name": "test",
    "protocol": "http",
    "loadBalancerPort": 80,
    "instancePort": 80
  }, 
  "nodes": []
}
    \end{minted}
  \end{center}
  \caption{Template including load balancer.}
  \label{list:loadbalancer}
\end{figure}



The feature of load balancer is implemented in \emph{cloudml-engine}, but at writing moment~(\date{April 2012}),
is not supported by jclouds.
The library, jclouds, have interfaces which suggest how to interact to create load balancers,
but full support of creating them is not yet complete.

Core design of introducing a load balancer into CloudML is expressed in \citelist{loadbalancer}.
The idea is to simply let every node within a template be bound to a given load balancer.
As a template is not bound to reflect a topology, but rather be part of one.

These interfaces in jclouds to create load balancers are utilized in \emph{cloudml-engine}, but not supported.
The main point about implementing this was that when jclouds fully support load balancers the dependency version
can be updated to the latest version, which would in theory give full support for this new feature.

\paragraph{\myac{SSH} keys.}

The current version of CloudML does not express how SSH keys should be defined.
In \emph{cloudml-engine} this is solved to use the default approach by jclodus library,
\ie for \myac{AWS} keys are generated as provisioning takes place.

Examples of solutions could be to include credentials in the template file, or the account file.
The authentication method between providers differ, \eg in \myac{AWS} SSH keys are assigned to a node,
while in Rackspace a root password is returned for each node.
To implement a common solution \emph{cloudml-engine} could use this password to automatically 
log into each node, inject a given \myac{SSH} key, obscure root password by changing it and log out.

\paragraph{Refactoring of actor logic.}

Logic for provisioning is currently located in \texttt{cloud-connector} module.
The reason for this is because the implementation built so it can differ between which library is used for provisioning, 
which currently is jclouds.

What can be done as an improved alternative is to move the provisioning logic out of \texttt{cloud-connector} 
module and into the actor classes~(\texttt{RuntimeInstance}).
The modularity of not being bound to a specific library is still important, and this must be abstracted away.
In the end the \texttt{repository} module would become increasingly more complex in term of pure construction.
The advantage of this would become clear as the complexity of the bridge between \emph{cloudml-engine}
and providers increases, \eg if features such as live managing should be introduced.

\section{Long term}

\paragrapn{Full deployment.}

CloudML as presented in this thesis is designed and implemented to provision nodes over a multicloud environment.
This idea can in some simplified version be said to only create instances on a set of supported providers.
The truth is that CloudML fulfills more than just that, and even to conquer such task by itself is not a small feat.

Although in reality end users are looking for something more than just creating instances in the cloud,
they eventually want to move their application to the cloud environment.
And to handle this they want a library which does as much of this as possible,
from pressing a button to having an application up and running on the cloud.

What CloudML should struggle to achieve in future versions is more means to accomplish \emph{full deployments}.
To accomplish this there are several topics to address, such as
\begin{ii}
  \iitem third party software,
  \ittem operating system,
  \ittem package managers,
  \iitem authentication,
  \iitem communication between nodes.
\end{ii}
Additional software must be considered to assist in many of these topics, \eg Puppet to handle third party software and installations.

\paragraph{Live managing.}

After provisioning is com

Manually terminating nodes = managing after provisioning.
ref maderia.

Full deployment is planed for next version of CloudML.

\note{Live managing.}

