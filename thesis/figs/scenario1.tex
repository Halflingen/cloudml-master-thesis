\begin{figure}[tb]
  \begin{center}
    \subfigure[Template with nodes.] {
      \begin{tikzpicture}[scale=0.7, transform shape]
        \node (Template) [class, rectangle split, rectangle split parts=2] {
          \textbf{:Template}
          \nodepart{second}name=``template1''
        };
        \node (Node) [class, rectangle split, rectangle split parts=2, right=of Template] {
          \textbf{:Node}
          \nodepart{second}name=``node1'' \\ cores=2 \\ disk=500
        };

        \draw[line] (Template) -- (Node);
      \end{tikzpicture}
      \label{fig:scenario1-1}
    }
    
    \subfigure[Instance.] {
      \begin{tikzpicture}[scale=0.7, transform shape]
        \node (Instance) [class, text width=5cm, rectangle split, rectangle split parts=2] {
          \textbf{:Instance}
          \nodepart{second}name=``node1'' \\ cores=2 \\ disk=500 \\ templateName=``template1''
        };
      \end{tikzpicture}
      \label{fig:scenario1-2}
    }
  \end{center}
  \caption{Object diagram of scenario with one node.}
  \label{fig:scenario1}
\end{figure}

